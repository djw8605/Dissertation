\chapter{Introduction}

\section{Campus Batch Computing}

A computational grid is a hardware and software infrastructure that provides dependable, consistent, pervasive, and inexpensive access to high-end computational capabilities\cite{foster2004grid}.  A campus grid is a specialized grid where resources are owned by the same organization, though can be in multiple administrative domains.  For our discussions of computation, we restrict our considerations to those campuses that have multiple computational resources.

A campus grid has become necessary to spread demand across multiple clusters.  This is important when demand for a single cluster is large, due to improved performance or more storage, and demand is low on other available clusters.  The campus grid can move computation from the in demand cluster to other clusters, which can result in a shorter time to completion for the user's jobs.

A campus grid requires a framework to distribute jobs to multiple clusters in a campus grid.  My thesis \cite{weitzel2011campus} proposed a solution based on HTCondor \cite{litzkow1988condor}.  The solution required installation of a campus factory \cite{TODO: campus factory} on each cluster's login node.  Although this solution was efficient and fault tolerant, as shown in my thesis, it proved difficult since the installation was manual.  There users had to install HTCondor on both the login node and their submit node.  Also, the security setup was based on IP whitelists, which can be defeated with IP spoofing.  Therefore, I set out to correct these deficiencies.

I have enhanced my masters thesis' solution to include:
\begin{itemize}
\item Easier installation through automation
\item Increased security through standard protocols
\item More supported cluster types and configurations
\item Access to computing through language frameworks
\end{itemize}

The new solution is named Bosco \cite{TODO}.  It uses secure protocols to connect to remote clusters in order to transfer files and submit / monitor jobs.  Installation of Bosco on remote clusters and the submit host has been automated with simple tools.  Clusters with restrictive firewalls are supported by multiplexing operations through a single secure connection.  Further, many cluster schedulers are supported by the underlying technology.  Access to Bosco and the computing environment has been enabled through programming language frameworks.





Most major research campuses, whether a university campus, or a national lab campus, have a research computing resource.  The computing resources are broken into two categories:

\begin{itemize}

\item Condominium - Resources are purchased by research groups for their dedicated use.  They are added to a cluster that may share infrastructure such as a filesystem or an interconnect.
\item Shared resources -  Resources are purchased by a central authority that are shared between multiple research groups.

\end{itemize}





\section{Data Transfers on Campus}

Data has become a limiting factor for scientific computing.  For batch computing, the issue is usually getting the data to the execution.  

As the users spread their computation across multiple clusters either on the campus, or across campus's, data distribution and collection becomes more difficult.  Before using the campus grid, a user would select a cluster which will do their processing.  The user then could host all of their data on that cluster by coping the data onto the cluster's shared filesystem.  The jobs could access the data from the distributed filesystem just as it would on the user's desktop, available from all executions at the same directory.


These assumptions do not hold for a campus grid.  A grid is made up of multiple computational clusters, with potentially many separate file systems.  There is no single filesystem that a user can access from every computational resource.  Therefore, the data must be handled differently than when on a shared filesystem.  

The data must flow from the data source to the computational resource in an efficient manner.  For this reason, it is necessary extract information about the data from the user, such as if it is shared between many jobs, or unique to each and every job.  Further, some data needs to be protected, therefore we must learn if the data is private or public.  Different transfer methods are necessary for different data types.  For example, if the data is shared between many executions, then it may be cached or can be group transferred.  

Some of these job attributes can be extracted from the job submission files without explicit instructions.  Input files that are shared between multiple executions can be categorized as shared.

\section{Contributions}

My dissertation will contain the following contributions:

\begin{itemize}

\item A framework for job submission to remote resources for which the user does not control.  Typical grid submission is done to custom interfaces such as the Globus Resource Allocation Manager \cite{foster1999globus} (GRAM), which are previously installed by an administrator.  Opportunistic resources, which are abundant, typically do not have specialized grid software installed.  Further, grid submission software adds administrator overhead for maintenance.  I propose a framework that does not require administrator intervention for remote submission to opportunistic resources.  The framework used interfaces that are installed on nearly all clusters that are typically used for interactive access.  It automates the submission and error handling of jobs submitted to remote resources, while providing the user a consistent interface over multiple, load balanced clusters.

\item A modern flexible policy language for describing data handling for computation coordination.  Users of grid submission software have to explicitly define how their files will be transferred from their submission host to the remote execution resource where the data will be consumed.  They have to coordinate with the storage resources and the computation on their own.  I propose a policy language that allows the scheduler to decide a appropriate method for data transfer.  It determines the transfer method by parsing a user given policy language for the data, such as if it is shared between multiple jobs, and if it is private data, and from the remote execution resource's capabilities and preferences.  This can be used to hide data transfer details from the user while optimizing data transfers between the submission host and the execution resources.

\item A new method of data transfer that will coordinate with the policy language described above and optimize for minimizing duplicated data transfers.  In this thesis, I propose to investigate transfer techniques that support multiple administrative and network domains that are typical in a distributed computing environment.

\end{itemize}

\section{Overview of Dissertation}

This dissertation is broken down into multiple chapters.  In Chapter \ref{chapter:relatedwork}, I will discuss related work in regards to my research.  In Chapter \ref{chapter:campusjobs} we will discuss how computing can be managed on the campus using the Bosco manager.  This chapter expands on my masters thesis.  Chapter \ref{chapter:campusdata} discusses methods for describing and moving data on the campus.  Chapter \ref{chapter:coordinatingstorage} describes a method for coordinating storage with computing using a policy language for data transfers, and includes a novel transfer technique for large shared data.




