\label{chapter:coordinatingstorage}

% re-iterate the introduction
In this dissertation we optimized distributed computing workflows on a campus grid.  We were interested in optimizing a researcher's use of the computational and storage resources on the campus to increase the reliability and decrease the time to solution for scientific results.  We first extended prior work to enhance the computational capabilities of researchers on a campus.  We then expanded our work to the data needs of modern workflows on the campus.

% Conclusion of BOSCO
Bosco is used to effortlessly create a remote submission endpoint on a cluster without requiring the administrator to install any software.  Bosco is a remote submission framework based upon HTCondor.  It uses the SSH protocol to submit and monitor remotely submitted jobs.  Additionally, it performs file transfers using the same SSH connection.

Improving the user experience was a primary goal of Bosco.  We addressed the user experience by improving the interaction with the user during the installation / configuration.  Another problem area we found is when a user must debug issues with distributed software.  In order to address this, we created a traceroute like utility.  The traceroute utility tests every step of the job submission process, from network access to a properly configured remote scheduler.  If an error is found at any step of the traceroute, a useful message is given to the user, including possible steps to fix the problem.

Bosco and the Campus Factory combine to make an easy to use framework that can distribute jobs to many computational clusters on a campus.  Users are able to effectively distribute their processing to multiple clusters using this framework.  I showed that Bosco transparently and effectively distributes computational jobs across multiple clusters on a campus, while maintaining simple usage for users.

Bosco's usage has increased since I originally published the Bosco paper.  For example, it is heavily used by the University of Chicago in order to submit OSG processing to opportunistic resources around the country.  They find Bosco useful since it does not require the installation of any software on the remote cluster.  Additionally, it has been used in several publications by the CMS experiment when they have used opportunistic resources for data processing.

% Conclusion of the CacheD
For data distribution on the campus, we have presented the HTCondor CacheD, a framework to decrease the stage-in time for large shared input datasets.  Our experiments proved that the CacheD decreases stage-in time for these datasets.  Additionally, the transfer method that the CacheD used can significantly affect the stage-in time of the jobs.

The BitTorrent transfer method proved to be a efficient method to transfer caches from the originator to the execution hosts.  In fact, the transfer time for jobs did not increase as the number of distinct nodes requesting the data increased.  Any bottlenecks that surround the cluster are therefore irrelevant using the BitTorrent transfer method.  In addition, we found that the CacheD using BitTorrent transfer method out performed the popular HTTP transfer method on the Open Science Grid.

% Conclusion of the Policy Language
In a distributed computing system, independent agents are designed to act on behalf of entities such as users, hosts, or entire clusters.  A policy language must exist so that the entities can express their goals to the agents.  

I have designed a policy language based on the HTCondor ClassAds that can be used for expressing policy in data distribution.  I have identified three interaction points for caching agents and designed the semantics for their interaction.  The policy language has been implemented in the CacheD, where it has been tested in Chapter \ref{chapter:campusdatadistribution}. 


\section{Future Work}

\subsection{Debugging}
% Improved debugging for Bosco
Debugging in distributed computing has always been a challenge.  Many different systems working together can create barriers for message and error propagation.

Debugging in Bosco has always been difficult.  To alleviate some of the debugging burden, we created the \texttt{tracroute} utility described in Section \ref{sec:boscotraceroute}.

\subsection{Flexible Transfer Types}
% Addition of transfer types, and flexible system for defining transfer methods

\subsection{Co-Scheduling}

% Scheduling jobs with CacheD

% Not only where the cache is, but with the policy language, we can guess where the job will be.



% optimize black holes for bittorrent
Further investigation of slow transfers must be completed in order further optimize the BitTorrent transfers on the OSG.  A possible solution could be to give up on the transfer after some timeout, or if the transfer speed is too slow.  Though, this timeout and transfer speed thresholds would be difficult to set accurately.

