\chapter{Related Work}
\label{chapter:relatedwork}

\section{Distributed Batch Systems}

Several batch systems and grid schedulers are able to schedule tasks on execution resources.  Examples of cluster schedulers that are frequently used are PBS \cite{pbstorque}, \mbox{HTCondor} \cite{litzkow1988condor}, and SLURM \cite{yoo2003slurm}.  Each scheduler is very good at resource management within a single administrative domain.  But, each of these resource managers have very limited ability to send processing to remote resources, which are typically under a separate administrative domain.  PBS and SLURM can send jobs between clusters that run the same schedulers.  HTCondor also has the ability to send processing to other clusters running HTCondor and it can also transform jobs to the language of other schedulers such as PBS and SLURM.

Grid schedulers have become more popular as the number of resources has increased.  Examples of grid schedulers are OSGMM \cite{website:osgmm} and GlideinWMS \cite{sfiligoi2008glideinwms}.  These schedulers are able to send jobs to remote resources using grid protocols.  OSGMM performs a direct grid submission to the remote resources using the GRAM  \cite{foster1999globus} interface.  GlideinWMS also submits to the GRAM interface of the cluster, but provides an overlay of HTCondor daemons on top of remote resources.  The overlay presents a consistent HTCondor interface to the computing resources for ease of use.  


\section{Distributed Storage Access}

Distributed file systems have long had their own storage access methods.  An example of this is Hadoop \cite{white2012hadoop}, a popular distributed file and processing system.  The only method to access Hadoop storage is through the Hadoop protocol.  On the Open Science Grid, the primary access methods are through file system independent middleware such as the Storage Resource Manager (SRM) \cite{shoshani2002storage} and XRootd \cite{dorigo2005xrootd}.  They provide a translation layer from system independent grid protocols and security mechanisms to the underlying storage system, such as Hadoop.  The Storage Resource Manager (SRM) is a previously popular protocol to access remote distributed filesystems.  It is a standardized protocol that allows remote, distributed access to large storage with APIs to balance transfers among many data servers.


Beyond storage access methods are storage schedulers.  These schedulers do not define a protocol to access the storage, rather they coordinate the access.  NeST \cite{bent2002flexibility} is a software-only grid aware storage scheduler.  It supports multiple transfer protocols into a storage device, including GridFTP \cite{allcock2005globus} and NFS \cite{walsh1985overview}.  Further, it provides features such as resource discovery, storage guarantees, quality of service, and user authentication.  It is layered over a distributed filesystem to provide access to it.  NeST functions as the interface and access scheduler for a storage device.  Features such as the storage guarantees and quality of service require NeST to be the only interface into the storage device, a very rare feature in today's grid storage.  Today's storage elements, such as the 3PB storage at Nebraska, include multiple interfaces to access the storage element.  Nebraska runs at least four \cite{attebury2009hadoop} methods of accessing and modifying storage, SRM \cite{shoshani2002storage}, GridFTP, XrootD, and Fuse \cite{szeredi2010fuse} mounted Hadoop.  All of these methods are required for compatibility with different access patterns and clients.  NeST could implement each of these protocols, but it would be extremely difficult to manage the storage centrally.  For example, Fuse is mounted on all 300 worker nodes.  The GridFTP and XrootD servers run on 10's of servers, with an aggregate bandwidth of 10Gbps.  Scaling quality of service and storage allocation / enforcement across all of these access methods would likely prove impossible.



\section{Data Transfer Mechanisms}

A popular consumer transfer method, Bittorrent, has been used for data transfer in computational grids by Wei et al \cite{wei2005collaborative, wei2005scheduling, wei2007towards}.  It has been shown to improve data transfer speeds when compared to traditional source and sink transfer methods, in this case FTP \cite{postel1985file}, which is very similar in architecture to GridFTP, a grid enabled FTP protocol.  The researchers did not compare performance of the bittorrent protocol when compared to modern grid transfer techniques, such as using HTTP local caching.  Further, the authors do not test bittorrent transfers across network partitions that are common on the grid.  For example, a worker node from one cluster may not be able to communicate directly with another cluster.  Therefore, bittorrent may not work between clusters, but will work inside clusters.

Globus online \cite{foster2011globus} is a web interface for transferring files between sites and sharing data with other users.  It offers an intuitive web interface for bulk transfers between endpoints.  It only supports the gridftp \cite{allcock2005globus} transfer protocol, and requires GridFTP implementations at all endpoints.

Stork \cite{kosar2004stork} is a data placement scheduler.  It can schedule data placement and transfers to and from remote storage systems.  Stork is innovative in that it treats data transfers similar to jobs.  It will queue transfers, and checks for proper completion of the transfers.

Kangaroo \cite{thain2001kangaroo} is another storage scheduler multi-level file access system.  It allows for multiple levels of staging in order to send job output back to a storage device.  It can do this by asynchronously staging data through multiple storage devices on it's path to the destination filesystem.  The Kangaroo system only addresses output data.

\section{Data Management}

Data management is different from data transfer and access in that it provides services on top of the file systems, such as meta-data storage and search capabilities.  An example of a Data management service is iRods \cite{rajasekar2010irods}.  It provides meta-data storage and querying, and rule based placements.  Further, it can handle transfers to storage resources.

DQ2 \cite{branco2008managing} and Phedex \cite{rehn2006phedex} are production transfer services for the Atlas and CMS physics experiments, respectively.  They are used to manage distributed transfers to and from sites inside the collaborations.  Additionally, they have had databases built on top of them that provide features such as combining files into datasets for easier bulk transfer management.  Both were designed for their experiments, and therefore would be very difficult to generalize for outside users.



