%%
%% This is file `skeleton.tex',
%% generated with the docstrip utility.
%%
%% The original source files were:
%%
%% nuthesis.dtx  (with options: `skeleton')
%% 

%%
%% For common degrees, you can use the class options:
%% phd, edd, ms, ma
%% phd is the default
\documentclass[print,phdproposal]{nuthesis}

\usepackage{graphicx}
\usepackage{listings}
\usepackage{hyperref}


\begin{document}
%% Start formatting the first few special pages
%% frontmatter is needed to set the page numbering correctly
\frontmatter

\title{Enabling Distributed Scientific Computing on the Campus}
\author{Derek Weitzel}
\adviser{Dr. David Swanson}
\adviserAbstract{Dr. David Swanson}
\major{Computer Science}
\degreemonth{Feburary}
\degreeyear{2014}
%%
%% For most people the defaults will be correct, so they are commented
%% out. To manually set these, just uncomment and make the needed
%% changes.
%% \college{Your college}
%% \city{Your City}
%%
%% For most people the following can be changed with a class
%% option. To manually set these, just uncomment the following and
%% make the needed changes.
%% \doctype{Thesis or Dissertation}
%% \degree{Your degree}
%% \degreeabbreviation{Your degree abbr.}
%%
%% Now that we know everything we need, we can generate the title page
%% itself.
%%
\maketitle
%%
%% You have a maximum of 350 words for your abstract, which includes
%% your title, name, etc.
%%
%% Required
\begin{abstract}
Distributed computing has evolved to include many federated resources across geographical, networking, and administrative boundaries.  The ability to access and use federated storage has lagged behind the ability to access the compute capabilities of the resources.  In order to incorporate storage resources, the storage first needs to be accurately described to meet the needs of the application.  Further, in order to optimize data movement, the data movement should be negotiated between the storage resource and the target.  The negotiated data movement device will allow the storage (or an agent on the behalf of the storage) to determine it�s preferred transfer methods, and to match that method with the target device, allowing each to express preferences and requirements.  In order to provide an additional option for distributed storage and data movement, a transfer method is introduced that is designed to run on federated resources by using peer to peer connections and data transfers.  To conclude, a variety of workflows are discussed and demonstrated using these enhanced tools with decreased time to completion.
\end{abstract}

%% Optional
%% \begin{copyrightpage}
%% \end{copyrightpage}

%% Optional
%% \begin{dedication}
%% \end{dedication}

%% Optional
%% \begin{acknowledgments}
%% \end{acknowledgments}

%% Optional
%% \begin{grantinfo}
%% \end{grantinfo}
%% The ToC is required
%% Uncomment these if need be

%% The ToC is required
\tableofcontents
%% Uncomment these if need be
% \listoffigures
% \listoftables
%%
%% ``Real'' beginning of the document.
%% mainmatter is needed to set the page numbering correctly
%%   mainmatter is needed after the ToC, (LoF, and LoT) to set the
%%   page numbering correctly for the main body
\mainmatter

%% Thesis goes here

\chapter{Introduction}

In this dissertation, we optimize distributed computing workflows on a campus grid.  We are interested in optimizing a researcher's use of the computational and storage resources on the campus to increase the reliability and decrease the time to solution for scientific results.  We first extend prior work to enhance the computational capabilities of researchers on a campus.  We then expand our work to the data needs of modern workflows.

\section{Campus Grid Computing}

The increase of performance of computer hardware following Moore's law \cite{schaller1997moore} has allowed scientists to tackle larger problems.  As they increase their use of research computing, their applications far exceed the locally available resources.  Such applications often turn to distributed computing to aggregate more computational, memory, or storage resources than locally available resources can provide.

%Despite the ever increasing performance of computer hardware following Moore's law \cite{schaller1997moore}, applications continue to keep pace with hardware's capabilities as researchers tackle larger problems.  For some users, their applications far exceed the capabilities of computers that are immediately available to them.  Such applications may be able to use multiple computers to aggregate  more computational, memory, or storage resources than a single computer can \mbox{provide}.

Batch computing can combine the computational, memory, and storage resources of multiple computers in a single cluster through concurrent scheduling of applications.  A computational grid is an extension of batch computing, where resources may be combined from multiple pools of resources to be used for an application.

A computational grid is a hardware and software infrastructure that provides dependable, consistent, pervasive, and inexpensive access to high-end computational capabilities \cite{foster2004grid}.  A campus grid is a specialized grid where multiple resources are owned by the same organization, although it may be in multiple administrative domains.  
%For our discussion of computation, we restrict our consideration to those campuses that have multiple computational resources.

A campus grid has become necessary to spread demand across multiple clusters.  This is important when demand for a single cluster is large, due to improved performance or increased storage, and demand is low on other available clusters.  One aim for a campus grid is to move computation from the in-demand cluster to other clusters, which can result in a shorter time to completion for the users' jobs.

To succeed, a campus grid requires a framework to distribute jobs to multiple clusters in a campus.  In \cite{weitzel2011campus}, I proposed a solution based on HTCondor \cite{litzkow1988condor}.  The solution required installation of a campus factory \cite{website:campusfactory} on each cluster's login node.  An on-demand overlay was created that could efficiently run high throughput jobs on multiple campus resources.  

Although my solution was efficient and fault tolerant, it was deficient in several ways.  Installation and setup of the campus factory was difficult since it was not automated.  The communication inside the overlay was insecure.  We set out to correct these deficiencies.

%Users had to install HTCondor on both the cluster's login node and the user's submit node.  Also, the security setup was based on IP whitelists, which can be defeated with IP spoofing.  Therefore, we set out to correct these deficiencies.

We have enhanced my Masters thesis' solution to include:
\begin{itemize}
\item Easier installation through automation
\item Increased security through secure key exchange
\item More supported cluster types and configurations
\item Improved access to computing through language frameworks such as R \cite{team2005r}
\end{itemize}

We created a framework for job submission to remote resources that the user does not control.  Typical grid submission uses custom interfaces such as the Globus Resource Allocation Manager (GRAM) \cite{foster1999globus}, which is previously installed by an administrator.  We assume resources do not have such dedicated grid software installed.  This framework does not require administrator intervention for remote submission to opportunistic resources.  The framework uses interfaces that are installed on nearly all clusters that are typically used for interactive access.  It automates the submission and error handling of jobs submitted to remote resources, while providing the user a consistent interface over multiple, load-balanced clusters.

The new framework is named Bosco \cite{weitzel2014accessing}.  It uses secure protocols to connect to remote clusters in order to transfer files and submit/monitor jobs.  Installation of Bosco on remote clusters and the submit host has been automated with simple tools.  Clusters with restrictive firewalls are supported by multiplexing operations through a single secure connection.  Furthermore, many cluster schedulers are supported by the underlying technology.  A diagram of the architecture of Bosco is shown in Figure \ref{fig:introboscoarch}.

\begin{figure}[h!t]
	\centering
	\includegraphics[width=\textwidth]{images/ArchitectureGraph1.pdf}
	\caption{Bosco Architecture}
	\label{fig:introboscoarch}
\end{figure}

Bosco, in coordination with technologies in HTCondor, enable a job distribution method which is provisioned based on demand.  A default Bosco installation is able to submit to one local cluster.  If that cluster does not meet the user's computational needs, then Bosco can be configured to submit to multiple clusters with load balancing between them.  If the user's computational needs are still not met with multiple clusters, they can configure Bosco to submit resource requests to national cyberinfrastructure such as the Open Science Grid (OSG) \cite{pordes2007open}.  The provisioning capabilities of Bosco creates an ever expanding network of available resources. The goal is to provide an expanding network of resources as shown in Figure \ref{fig:boscogrowing}.

\begin{figure}[h!t]
	\centering
	\includegraphics{images/BoscoGrowing.pdf}
	\caption{Bosco's Growing Reach as Demand Increases}
	\label{fig:boscogrowing}
\end{figure}

In order to ease access to Bosco for data processing, an interface has been developed in the most widely used data processing language, R.  This BoscoR framework enables users to never leave their R environment in order start remote data processing.

But Bosco is not enough for researchers that have large data requirements.  Input and output data are explicitly listed by the user.  The data is transferred over the secure, but slow, connection between the submitter and resource for every job.  Therefore, we must consider data and storage management on the campus grid.


%Most major research campuses, whether a university campus, or a national lab campus, have a research computing resource.  The computing resources are broken into two categories:

%\begin{itemize}

%\item Condominium - Resources are purchased by research groups for their dedicated use.  They are added to a cluster that may share infrastructure such as a filesystem or an interconnect.
%\item Shared resources -  Resources are purchased by a central authority that are shared between multiple research groups.

%\end{itemize}





\section{Data Management on Campus}

There are many challenges in data management and distribution in scientific computing \cite{deelman2008data}.  For batch computing, one challenge is transferring the data from the user's computer to the execution resources.  Large data workflows can strain the network near the data's source, which can result in unreasonable amounts of batch time used solely for data transfer.

Data management is the framework and policies controlling data through the research cycle.  In this dissertation, we are concerned with optimizing data management when using campus computational resources.

As users spread their computation across multiple clusters either on the campus or across campuses, data distribution and collection becomes more difficult.  Before using the campus grid, a user would select a cluster to do their processing.  The user then could host all of their data on that cluster by copying the data onto that cluster's shared filesystem.  The jobs access the data from the shared filesystem just as it would on the user's desktop, available for all executions at the same directory.

These assumptions do not hold for a campus grid.  A grid is made up of multiple computational clusters, with potentially many separate filesystems; no single filesystem is accessible access from every computational resource.  Further, the shared filesystem could become a bottleneck if many jobs are requesting the same data simultaneously.  Therefore, data management techniques must evolve along with computation.  


Most distributed batch schedulers are able to transfer the input data for each job execution.  Each job starts with an empty execution area and the scheduler will transfer the files into it.  When the user is not using the scheduler to transfer data, the input data must still reach the execution host.  Data will be transferred from the source (usually the user's computer) to the execution resources for processing.  The network connection between the source and the execution resources may be a bottleneck for the computation.  Frequent re-transfers of the same input data will further congest the network between the source and the execution resources.

In this dissertation, we optimized two attributes of distributed data management: efficient transfer methods and reduction of duplicate transfers.

We introduce the CacheD \cite{weitzel2015pdpta}, a caching and data transfer daemon for input data in distributed computing.  The CacheD uses novel data management methods based on technology developed for large peer-to-peer data transfers on the Internet, BitTorrent.  It also caches input data on the execution resources to enable quick transfers on subsequent requests for the same input data.

Similar to the work with Bosco, the CacheD does not require privileged access in order to provision storage resources.  It can use the storage on worker nodes spread across multiple clusters as a data input caching system.

\section{Data Distribution Policy Language}

Users of grid submission software currently have to describe how their files will be transferred from their submission host to the remote execution resource where the data will be processed.  They have to coordinate the storage and computational resources without help.  We propose a policy language that allows an agent to decide an appropriate method for data transfer.  It determines the transfer method by negotiating between the following three sources: a user-given policy language for the data, the remote execution resource's capabilities and preferences, and the submitting resource's capabilities and preferences.  In addition, the policy language should determine if the cache should be replicated to multiple resources.  A modern flexible policy language for describing data distribution for campus users is needed.

In addition to the CacheD described above, a policy language must be made in order to help the CacheD make decisions when interacting with other agents, such as other CacheDs or the local node.  We discuss extending this policy language to include custom attributes that users can include to improve choices on data distribution.

The policy language utilizes the \mbox{ClassAds} \cite{raman1998matchmaking} language.  These \mbox{ClassAds} were originally developed in the context of matchmaking between computational resources and potential jobs.  ClassAds are a schema-free language for describing heterogeneous resources.  We demonstrate usage with new attributes that pertain to storage and expressions that can be evaluated to make decisions.

The user must specify preferences for the cache to consider.  Examples include: where should this cache be distributed, how should the cache be distributed, and how long the cache should be stored.  Each of these preferences must be negotiated with the preferences of the CacheD that may store or is storing the cache.  The user's preferences will affect how fast the cache is transferred (different transfer methods are more efficient than others) and also, on which and how many nodes that cache should be replicated.

Further, each CacheD must coordinate with one another in order to distribute the caches in an efficient method.  Replication of caches between CacheDs must be negotiated.  A CacheD may decide, through evaluating its own policies, whether or not to accept a cache to be stored.  These policies are again expressed in the ClassAd policy language.



\section{Overview of Dissertation}

This dissertation describes how data intensive applications can be run in a distributed campus environment.

\begin{description}
	\item[Chapter \ref{chapter:relatedwork}:]  There are many distributed computing platforms available publicly.  In this chapter, we will discuss these schedulers and differentiate them with Bosco.  Also, we will discuss other available data management, distributed storage, and caching systems.
	
	\item[Chapter \ref{chapter:campusjobs}:] We will discuss how computing can be managed on the campus using the Bosco framework.  We will also discuss a case study of integrating Bosco with the programming language, R, in order to provide an easy-to-use interface to campus distributed computing.
	
	\item[Chapter \ref{chapter:campusdatadistribution}:] We will discuss the CacheD, a campus data distribution service.  The CacheD is able to combine novel transfer methods with data caching to improve the stage-in time for large data sets.  Through evaluation, we show that the CacheD demonstrates a significantly shorter stage-in time for large data sets over existing solutions that have been deployed.
	
	\item[Chapter \ref{chapter:campusstoragepolicylanguage}:]  
	Simply caching and transferring data does not provide the flexibility that the CacheD requires to operate in a distributed environment.  In this chapter, we will discuss the policy framework and language that enable the CacheD to interact with the user and other daemons in order to make decisions.
\end{description}


\newpage
\section{A Note on Terms}
High performance and distributed computing often use terms inconsistently.  Below is a definition list of such terms, and how we will define them for this dissertation:


\begin{description}
	\item[Job:] A packaged unit of work with input and output.  A job may consume computational, memory, network, and/or storage resources in a batch system.
	\item[Workflow:] A logical grouping of jobs executed on resources.  The jobs may have some ordering.
	\item[Campus:] An organization that may have multiple administrative domains which may vary access policies to resources.
	\item[Execution Resource:] A resource which fulfills the requirements of a job and may also run it.  This may be a worker node in a cluster.
	\item[Cluster:] A set of execution resources that have high interconnection bandwidth and are managed by a single scheduler.
	\item[Batch System:] A scheduler for the resources of a cluster.
	\item[Agent:] An independent entity that can make decisions on its own without the control of another entity.  In this dissertation, we will use the word agent to describe a daemon which can make independent decisions without the explicit control of other daemons.
	\item[Pilot:] Pilot jobs are containers that once started, will request work from the user's job queue.
\end{description}







\chapter{Related Work}
\label{chapter:relatedwork}

\section{Distributed Batch Systems}


Several batch systems and grid schedulers are able to schedule tasks on execution resources.  Examples of cluster schedulers that are frequently used are PBS \cite{pbstorque}, \mbox{HTCondor} \cite{litzkow1988condor}, and Slurm \cite{yoo2003slurm}.  Each scheduler is very good at resource management within a single administrative domain.  Each of these resource managers has a very limited ability to send processing to remote resources, which are typically under a separate administrative domain.  PBS and Slurm can send jobs between clusters that run the same schedulers.  HTCondor also has the ability to send processing to other clusters running HTCondor, and it can also transform jobs to the language of other schedulers such as PBS and Slurm.

Grid schedulers have become more popular as the number of resources has increased.  Examples of grid schedulers are OSGMM \cite{website:osgmm} and GlideinWMS \cite{sfiligoi2008glideinwms}.  These schedulers are able to send jobs to remote resources using grid protocols.  OSGMM performs a direct grid submission to the remote resources using the GRAM  \cite{foster1999globus} interface.  GlideinWMS also submits to the GRAM interface of the cluster, but provides an overlay of HTCondor daemons on top of remote resources.  The overlay presents a consistent HTCondor interface to the computing resources for ease of use.  


\section{Distributed Storage Access}

Distributed file systems have long had their own storage access methods.  An example of this is Hadoop \cite{white2012hadoop}, a popular distributed file and processing system.  The only method to access Hadoop storage is through the Hadoop protocol.  On the Open Science Grid, the primary access methods are through file system independent middleware such as the Storage Resource Manager (SRM) \cite{shoshani2002storage} and XRootd \cite{dorigo2005xrootd}.  They provide a translation layer from system independent grid protocols and security mechanisms to the underlying storage system, such as Hadoop.  The Storage Resource Manager (SRM) is a previously popular protocol to access remote distributed filesystems.  It is a standardized protocol that allows remote, distributed access to large storage with APIs to balance transfers among many data servers.


Beyond storage access methods are storage schedulers.  These schedulers do not define a protocol to access the storage, rather they coordinate the access.  NeST \cite{bent2002flexibility} is a software-only grid aware storage scheduler.  It supports multiple transfer protocols into a storage device, including GridFTP \cite{allcock2005globus} and NFS \cite{walsh1985overview}.  Further, it provides features such as resource discovery, storage guarantees, quality of service, and user authentication.  It is layered over a distributed filesystem to provide access to it.  NeST functions as the interface and access scheduler for a storage device.  Features such as the storage guarantees and quality of service require NeST to be the only interface into the storage device, a very rare feature in today's grid storage.  Today's storage elements, such as the 3 petabyte storage at University of Nebraska, include multiple interfaces to access the storage element.  Nebraska runs at least four methods of accessing and modifying storage \cite{attebury2009hadoop}, SRM \cite{shoshani2002storage}, GridFTP, XrootD, and Fuse \cite{szeredi2010fuse} mounted Hadoop.  All of these methods are required for compatibility with different access patterns and clients.  NeST could implement each of these protocols, but it would be extremely difficult to manage the storage centrally.  For example, Fuse is mounted on all 300 worker nodes.  The GridFTP and XrootD servers run on 10s of servers, with an aggregate bandwidth of 10 Gbps.  Scaling quality of service and storage allocation/enforcement across all of these access methods would likely prove impossible.



\section{Data Transfer Mechanisms}

A popular consumer transfer method, BitTorrent, has been used for data transfer in computational grids by Wei, Fedak, \& Capello \cite{wei2005collaborative, wei2005scheduling, wei2007towards}.  It has been shown to improve data transfer speeds when compared to traditional source and sink transfer methods, in this case FTP \cite{postel1985file}, which is very similar in architecture to GridFTP, a grid enabled FTP protocol.  The researchers did not compare performance of the BitTorrent protocol when compared to modern grid transfer techniques, such as using HTTP caching.  Further, the authors did not test BitTorrent transfers across network partitions that are common on the grid.  For example, a worker node from one cluster may not be able to communicate directly with another cluster.  Therefore, BitTorrent may not work between clusters but will work inside clusters.

Globus Online \cite{foster2011globus} is a web interface for transferring files between sites and sharing data with other users.  It offers an intuitive web interface for bulk transfers between endpoints.  It only supports the GridFTP \cite{allcock2005globus} transfer protocol and requires GridFTP implementations at all endpoints.

There are also popular data transfer tools used on clusters such as secure copy (SCP) from OpenSSH \cite{openssh} and rsync \cite{rsynce}.  SCP is a simple copy tool that uses the Secure Shell (SSH) protocol to transfer files from a source to a client.  Rsync is also able to copy files from a source to a client, but it can also do differential copies, where only the changed portion of a directory will be copied at a time.  Both of these methods are used heavily when the data is small.  But they both use single stream TCP in order to transfer data, which has been shown to be slower than multi-stream TCP which is used in GridFTP \cite{allcock2005globus} or BitTorrent.

\section{Data Management}

There have been previous policy frameworks for distributed storage.  These frameworks have largely been designed to move data between a few large filesystems.  Therefore, the interactions are rare but could make significant changes to the system.  In contrast, the CacheDs have frequent interactions with other agents, but each one has a minimal impact on the entire system.

Data management is different from data transfer and access in that it provides services on top of the file systems, such as meta-data storage and search capabilities.  An example of a data management service is the integrated Rule-Oriented Data System iRODS \cite{rajasekar2010irods}.  It provides metadata storage, querying, and rule-based placements.  Also, it can handle transfers to storage resources.  When given input, iRODS also has the capability to create rules and take actions on data.  It creates a small policy framework that upon certain actions, can execute micro-services.  This iRODS policy framework is much more extensive than what we created in Chapter \ref{chapter:campusstoragepolicylanguage}.  Our framework is designed for frequent interaction between many agents acting independently.  The rules for iRODS can be large and cumbersome for simple data replication.  Further, to do anything substantial with the rules, custom code must be written.

Stork \cite{kosar2004stork} is a data placement scheduler.  It can schedule data placement and transfers to and from remote storage systems.  Stork is innovative in that it treats data transfers similar to jobs.  It will queue transfers and check for proper completion of the transfers.

Kangaroo \cite{thain2001kangaroo} is another storage scheduler multi-level file access system.  It allows for multiple levels of staging in order to send job output back to a storage device.  It can do this by asynchronously staging data through multiple storage devices on its path to the destination filesystem.  The Kangaroo system only addresses output data.

DQ2 \cite{branco2008managing} and Phedex \cite{rehn2006phedex} are production transfer services for the Atlas and CMS physics experiments, respectively.  They are used to manage distributed transfers to and from sites inside the collaborations.  Additionally, they have had databases built on top of them that provide features such as combining files into datasets for easier bulk transfer management.  Both were designed for their experiments, and therefore, would be very difficult to generalize for outside users.

Distributed filesystems such as Hadoop also provide a small amount of data placement policy that can be configured.  For example, Hadoop can be configured to replicate the contents of a directory at least $X$ times.  Further, a script can be given to Hadoop which it can query to create a topology of the data center, further providing control of how the data replicas are sent.  This topology script has been used by me to create a data center aware Hadoop replication policy \cite{he2012hog}.

\begin{figure}[ht!]
	\centering
	\includegraphics[width=\textwidth]{images/BackgroundStorageDiagram2.pdf}
	\caption{Background on Storage Technologies}
	\label{fig:backgroundstorage}
\end{figure}

Figure \ref{fig:backgroundstorage} shows an overview of the different protocols and services, and how they fit into the three categories: Data Management, Data Services, and Transfer Protocols.

\section{State of Practice in Campus Computing}

In order to illustrate the available technologies on the grid, we will begin with a typical use case.  We will then describe the technologies that could enable this computing on the grid.

\subsection{Use Case}
We will begin with a typical use case.  In this particular case, we will focus on the use of BLAST \cite{altschul1990basic}.  BLAST workflows typically include the following files:

\begin{itemize}
	\item Executable
	\item Database
	\item Query Files
\end{itemize}

Each of these files have different properties.  The executable is relatively small, maybe 10s of megabytes.  But it is shared between all executions of BLAST.  The query files are unique to each job but are typically very small, not exceeding 1 MB.  

The database is a large collection of proteins that is searched for each protein in the query file.  Many databases are publicly available for use in BLAST.  The most common is the non-redundant (NR) database, which is currently 50 GB and updated weekly.

\subsection{Current Approach}

If the user has a BLAST application and wishes to run numerous jobs, they must first gain access to computational resources.  They may have access to a campus cluster.  In that case, they will log into the campus cluster.  The first step for the user is to learn the scheduler language.  There are many different languages, such as PBS, Slurm, or LSF \cite{computinglsf}.  All of these languages have slightly different syntax.

Once the scheduler language has been learned enough to write a submission file, the data must be transferred to the cluster from their laptop.  This is usually accomplished with a tool such as SCP from the OpenSSH \cite{openssh} package.  This will be transferred slowly as SCP only uses a single encrypted stream to send data.  For the 50 GB NR database from their wireless connected laptop to the cluster could take two hours (assuming 54 Mbps wireless), if nothing goes wrong with the transfer.

Once the data is on the cluster, the user will submit the jobs to the scheduler to process the data.  The BLAST database will be copied for each and every execution to the execution resources from the cluster's shared filesystem.

Once the computation has completed, the user will copy the output data back to their laptop for further analysis.

\subsection{Issues with Current Approach}

There are many issues with the current approach that I will point out.

\begin{itemize}
	\item Users must learn one or more scheduler languages.  If users wants to submit to only one cluster, then they only need to learn one submission language.  But if their demands grow, and they need more resources, they will need to learn another programming language.
	\item Data copies are very expensive and should be minimized.  The NR database is updated frequently;  therefore, it must be updated on the cluster frequently.
	\item Once on the cluster, each and every worker node will need to copy the NR database in order to process it.  This copy will happen every time, as there is no caching in the vast majority of distributed filesystems.
\end{itemize}









\chapter{Campus Job Distribution}
\section{Introduction}

In this chapter I will discuss the methods developed to aid in distributed scientific computing on a research campus.  

\section{Access to Computation on the Campus}

This could be merged with the introduction

Discuss
\begin{itemize}
\item Typical Campus resources
\item How users get access to these resources
\end{itemize}

\section{Bosco}

\begin{itemize}
\item How Bosco addresses the concerns in the above.
\item Why users would want to use Bosco over other solutions
\item Architecture
\end{itemize}

\section{Load Balanced Access to Computational Resources}

\begin{itemize}
\item Advantages of using the Campus Factory
\item Architecture
\item How the campus factory + bosco create utopia!
\item How the submission works using bosco
\end{itemize}





\section{Conclusion}





\chapter{Campus Storage Access}
\section{Introduction}

Distributed computing has evolved to include many federated compute elements across geographically and administrative boundaries.  The ability to access and use federated storage has lagged behind the ability to access the compute capabilities of the resources.  In order to access the storage, the storage first needs to be accurately described to meet the needs of the application.  Further, in order to optimize data movement, the data movement should be negotiated between the storage element and the target.  The negotiated data movement device will allow the storage (or an agent on the behalf of the storage) to determine it�s preferred transfer methods, and to match that method with the target device, allowing each to express preferences and requirements.  In order to provide an additional option for distributed storage and data movement, a transfer method is introduced that is designed to run on federated resources by using peer to peer connections and data transfers.  


\section{Measuring Storage}
In order to provide matchmaking for resources, the resources need to be accurately described and advertised.  This will require measuring the storage capabilities and capacity of the resources and advertising those attributes to the matchmaking service.

The measurements would be performed on the execution target as well as against the storage targets.  The execution targets would measure the storage capabilities in order to determine if the jobs can run.  The storage targets would be measured in order to determine the number of jobs that could be run against a the target.


\subsection{Ranking Storage}
In order to find the most ideal resource for a job, the resources need to be ranked.  The simplest is a greedy approach where the resources are simply ranked by their benchmark speeds.  Additionally, they should only be ranked on the attributes requested by the job, IE, if the job is only requesting X iops, then only rank resources on the IOPS available.

It is not clear how the ranking should work.  If we assume that the user accurately describes their application�s needs, then we can pack the jobs onto resources by placing the job on the resource that meets the IOPS requirements, but has the least amount of IOPS remaining.  This will be an area of research to compare scheduling techniques on execution resources when considering their storage capabilities.


\section{Data Movement}
We will consider 3 different types of data.  The input data, output data, and the job sandbox.  The job sandbox is the environment from which the job will run.  The sandbox is important since the user designs their job to run in this sandbox, and it must be maintained in order for the job to run.  Also, the sandbox is usually identical for many executions of the program.

After finding a resource to run on, the job sandbox and input data must be transferred to the remote host.  In order to do this, the remote execution host and the submitter must negotiate how to get the data there.  For example, does the remote host have access to the same NFS server?  Can it mount it?
Description for these items
Logically, we can separate these items into 2 categories
\begin{itemize}
\item Requirements for the application
\item Acceptable methods of data movement
\end{itemize}

The users must specify these items in the description of their jobs.  The exact language used for these specifications is yet to be determined.  

The language for the requirements will be similar to the current specifications for memory and cpu.  The user will request a certain amount of storage parameters, and machines will need to provide these metrics just as they do now with cpu and memory.

The acceptable methods for data movement can either be specified by the user, or by the submitting system.  The system can stage the data to a third party, which will then be used for the transfer.  This can be especially useful if multiple jobs use the same input data, a useful example of this is HCC�s use of LVS to server common files on the OSG.  The server could automatically choose to use HTTP to transfer the files, especially since there are many common files, and the files would be cached on the remote sites using normal HTTP proxy caches.

Another possible scenario is when starting a job on Amazon EC2.  If it is a virtual machine job, then input data could be created as a CD drive, or a block device, and input into the machine using the block device as input storage.

\section{Policy language for matchmaking storage}
The goal is to enable the user to describe their application to the scheduler in such a way that the scheduler can make intelligent decisions on:
If the application can run on the pool
Where is the ideal location for the job to run
How to get the data to and from the application


\section{Defining Storage Target}
In this paper, we define storage based on it�s capabilities:
\begin{itemize}
\item The total space available for an application or set of applications to store data.
\item The bandwidth available to the storage target.
\item The IOPS available to to read / write to the storage (more applicable to local storage).
\item Access Protocol
\end{itemize}

Therefore, when mentioning storage, we must specify all of these attributes.






\section{Another Method for Data Transfer}
In addition to the above methods for transferring data to remote worker nodes, and specifying storage parameters, we can also provide another method for getting data to worker nodes that will better fit the current state of clusters and cyberinfrastructure.  The proposed methods is a dynamic deployment of a storage federation.  This can be done across a single cluster, across many clusters, or over an entire national infrastructure such as the OSG.

This new method for data transfer relies on peer to peer transfers.  Data is transferred from it�s peer rather than from a single host.  As with all peer to peer systems, the benefits from this method include decreasing the required bandwidth from any single source.  As well as lower latency transfers.


\chapter{Coordinating Campus Storage and Computation}

\label{chapter:coordinatingstorage}


% As a general rule, do not put math, special symbols or citations
% in the abstract
%\begin{abstract}

%A batch processing job in a distributed system has three clear steps, stage-in, execution, and stage-out.  As data sizes have increased, the stage-in  time has also increased.  In order to optimize stage-in time for shared inputs, we propose the CacheD, a caching mechanism for high throughput computing.  Along with caching on worker nodes for rapid transfers, we also introduce a novel transfer method to distribute shared caches to multiple worker nodes utilizing BitTorrent.  We show that our caching method significantly improves workflow completion times by minimizing stage-in time while being non-intrusive to the computational resources, allowing for opportunistic resources to utilize this caching method.


%\end{abstract}




\section{Introduction}

Large input datasets are becoming common in scientific computing.  Unfortunately for campus researchers, the staging time of the datasets to computational resources has not kept pace with the increase in dataset sizes.  The typical large dataset workflow may consist of thousands of individual jobs, each sharing input files.  

The campus resources made available to researchers are shared; therefore, the researchers have the limitation of not having access to install programs on the clusters.  Previous work \cite{weitzel2014accessing} built an overlay on top of campus resources to create a virtual, on-demand pool of resources for task execution.  We expand the capabilities of this virtual pool to include data caching and novel transfer methods to enable big data processing.

%Each node in the virtual pool will run multiple jobs, by default, all batch systems will transfer the large input file for each job.  Our goal is to minimize the number of times the input data is transferred from the submit host to the execution target.  Therefore a framework of local caching and distributed file transfer is proposed to address these situations.

An excellent example of a big data workflow is that of the bioinformatics application: BLAST \cite{altschul1997gapped}.  Each BLAST query requires an entire reference database, which can range in size from a few kilobytes to many gigabytes.  The workflow to run a BLAST query requires a large stage-in time in order to make the reference database available.  Additionally, the databases are frequently updated with new entries.

Users in the past have copied the database using various methods.  The na\"{i}ve method includes copying the database for each job.  Storing the database on a shared filesystem has the same effect as copying the database for each job, since the database must be transferred to the execution node for each job.  We propose caching the database on the node for subsequent executions.

We find that the BLAST workflow described above is common among large data researchers.   

Bosco \cite{weitzel2014accessing} is a remote submission tool that can create overlay virtual pools designed for campus resources.  In previous work, Bosco allowed campus researchers to submit high throughput jobs to high performance clusters.  We extend Bosco to include data caching and novel data transfer methods. 

We limit our design and analysis to a campus cluster computing environment.  Our solution is unique in that it is designed to run opportunistically on the campus computing resources.  Additionally, they do not require administrator intervention in order to create a virtual, on-demand pool of resources.


\section{Background and Related Work}

% Background on caching

Data caching on distributed systems has been used many times and at many levels.  Caching can be done on the storage systems and on the execution hosts, as well as well as in within the infrastructure separating the two.

Some distributed filesystems use local caches on the worker nodes.  GPFS \cite{schmuck2002gpfs} has a read-only cache on each worker node that can cache frequently accessed files.  It is designed for a fast, shared filesystem and is recommended when file access latency is a concern.  It is not recommended for large files since internal bandwidth to the local disk is assumed to be less than the bandwidth available to the GPFS shared filesystem.  GPFS file transfers are typically done over high speed interconnects which can provide high bandwidth for large files.  These interconnects are not typically available to a user's jobs for transferring data from a remote source.

% pcache?

% HTTP caching
HTTP caching is used throughout the web to decrease latency for page loads and to distribute requests among servers.  In high throughput computing, a forward proxy is commonly used to cache frequent requests to external servers.  The forward proxy caches files requested through it, and will respond to subsequent requests for the same file by reading it from memory or its own disk cache.

The HTTP forward proxy caching does have limitations.  The HTTP protocol was designed and is used primarily for websites.  Websites have very different requirements from high throughput computing.  The data sizes are much smaller.  Software designed as forward proxies, such as Squid \cite{squidcacheurl}, are optimized for web HTTP traffic, and therefore do not handle large data file sizes optimally.  Further, the Open Science Grid (OSG) \cite{pordes2007open} sites typically only have one or possibly a few squid caches available to user jobs.  They are not designed to scale to large transfers for hundreds of jobs, our target use case.

% chirp / parrot
Parrot \cite{thain2005parrot} is another application that will cache remote files when using certain protocols.  Parrot uses interposition \cite{thain2001multiple} to capture and interpret IO operations by an unmodified binary application.  The interposition allows Parrot to provide a transparent interface to remote data sources.  Parrot caches some of those sources such as HTTP with GROW-FS, a filesystem using HTTP.  Parrot caches an entire file to the local storage.  Parrot must download directly from the source the first time it is requested, exhausting WAN bandwidth quickly for large files.


% CernVM-FS
CernVM-FS \cite{blomer2011cernvm} provides a filesystem over the HTTP protocol.  It integrates into the worker node system using the FUSE \cite{szeredi2010fuse} interface.  The CernVM-FS local node client caches files on the node, as well as using Squid to cache files at the site.  Again, since it uses the HTTP, it's not designed to cache large files.  Neither the Squid caches nor the web servers optimally transfer large files, nor are they designed for large data sizes.  Further, CernVM-FS requires administrator access in order to install and configure, a privilege that campus users do not have.


% xrootd caching
XrootD \cite{dorigo2005xrootd} is designed for large data access, and it has even been used for WAN data transfers \cite{bauerdick2012using} using a federated data access topology.  There has been some work in creating a caching proxy for the XrootD \cite{bauerdick2014xrootd}.  The caching proxy is designed to cache datasets on filesystems near the execution resources.  The caching proxy requires installation of software and the running of services on the cluster.  Unprivileged campus users will be unable to run or install these services.



% transfer protocols
% HTTP

% Caching in the 
We define local caching as saving the input files on the local machine and making them available to local jobs.  Local caching is different from site caching, which is done in the OSG by Squid caches.  We define site caching as when data files are stored and available to jobs from a closer source than the original.  In most cases on the OSG, the site cache is a node inside the cluster that has both low latency and high bandwidth connections to all of the execution hosts.

%BitTorrent
We use distributed transfer to mean transfers that are not from a single source.  In our case, we will be using BitTorrent \cite{cohen2008BitTorrent}, in which a client may download parts of files from multiple sources.  Additionally, the client may make available to other clients parts of the files that have already been downloaded.

BitTorrent is a transfer protocol that is designed for peer-to-peer transfers of data over a network.  It is optimized to share large datasets between peers. The authors of \cite{wei2005scheduling} and \cite{wei2007towards} discuss scheduling tasks efficiently in peer-to-peer grids and desktop grids.  Their discussion does not take into account the network bottlenecks that are prevalent in campus cluster computing.  

In \cite{briquet2007scheduling}, the authors use scheduling, caching, and BitTorrent in order to optimize the response time for a set of tasks on a peer-to-peer environment.  They build the BitTorrent and caching mechanisms into the middleware which is installed and constantly running on all of the peers.  They do not consider the scenario of opportunistic and limited access to resources.  Their cluster size is statically set, and therefore may not see the variances that users of campus clusters may see.



\section{Implementation}



The HTCondor CacheD is a daemon that runs on both the execution host and the submitter.  For our purposes, a cache is defined as an immutable set of files that has metadata associated with it.  The metadata can include a cache expiration time, as well as ownership and acceptable transfer methods.

The CacheD follows the HTCondor design paradigm of a system of independent agents cooperating.  Each CacheD makes decisions independently of each other.  Coordination is done by CacheD�s communicating and negotiating with each other.

Each caching daemon registers with the HTCondor Collector.  The collector serves as a catalog of available cache daemons that can be used for replication.

% Talk about the transfer plugin
In addition to the CacheD, a transfer plugin is used to perform the cache transfers in the job's sandbox.  The plugin uses an API to communicate with the local CacheD to request local replication requests to the local host.  After the cache is transferred locally, the plugin then downloads the cache to the job's working directory.

Expiration time is is used for simple cache eviction.  A user creates a cache with a specific expiration time.  After a cache has expired, a caching server may delete it to free space for other caches.  The expiration may be requested to be extended by the user 

The CacheD supports multiple different forms of transferring data.  Using HTCondor's file transfer plugin interface, it can support pluggable file transfers.  For this paper, we will only use the BitTorrent and Direct transfer methods.  The BitTorrent method uses the libtorrent library to manage BitTorrent transfers and torrent creation.  The Direct method uses an encrypted and authenticated stream to transfer data from the source to the client.

An important concept of the caching framework is a cache originator.  The original daemon that the user uploaded their input files to is the cache originator.  The cache originator is in charge of distributing replication requests to potential nodes, as well as providing the cached files when requested.

The caching daemons interact with each other during replication requests.  A cache originator sends replication requests to remote caching daemons that match the replication policy that is set by the user.  The remote caching daemon then confirms that the cache data can be hosted on the server.  The remote cache then initiates a file transfer in order to transfer the cached data from the origin to the remote CacheD.

The receiving CacheD can deny a replication request for many reasons, including:
\begin{itemize}
\item The resource does not have the space to accommodate the cache.
\item The resource may not have the necessary bandwidth available in order to transfer the cache files.
\item The resource does not expect to be able to run the user's jobs and has determined that the cached files will not be used.
\end{itemize}

The ability of the receiving CacheD to deny a replication request follows HTCondor's independent agent model.

The policy expression language is modeled after the matchmaking language in the HTCondor system \cite{raman1998matchmaking}.  The caching daemon is matching the cache contents to a set of resources; therefore, it is natural to use HtCondor's same matchmaking language that is used to match jobs to resources.  Once a resource is determined to match the cache's policy expression, the caching daemon will contact the resource's caching daemon in order to initiate a cache replication.  The caching daemon on the remote resource is an independent agent that has the ability to deny a caching replication even after matchmaking is successful.  

Libtorrent is built into the CacheD to provide native BitTorrent functionality.  The CacheD is capable of creating torrents from sets of files in a cache, as well as downloading cache files using the BitTorrent protocol.  Since this is a distributed set of caches, we will not use a static torrent tracker.  Rather, we will use a Distributed Hash Table \cite{dinger2009decentralized} and local peer discovery \cite{legout2007clustering} features of the BitTorrent protocol.  This ensures that there are no single points of failure.



\subsection{Creation and Uploading Caches}
The user begins using the caching system by uploading a cache to their own CacheD, which then becomes the cache originator.  This is very similar to a user submitting a job to their own HTCondor SchedD.  Using the cache's metadata, the CacheD decides whether to accept or reject the cache.  If the CacheD accepts the cache, it stores the metadata into resilient storage.  The user then proceeds to upload the cache files to the CacheD.

The CacheD stores the cache files into it�s own storage area.  Once uploaded, the CacheD takes action to prepare the cache to be downloaded by clients.  This includes creating a BitTorrent torrent for the cached files.  

Numerous protections are used in order to ensure proper usage of the CacheD.  The upload size is enforced to the size advertised in the metadata.  The client cannot upload more data to the CacheD than was originally agreed upon during cache creation.  Further, the ownership of the cache is stored in the metadata, and is acquired by authenticating with the client upon cache creation.  Only the owner may upload and download files from the cache directly.

A client may mark a cache as only allowing certain replication methods.  This can be useful if a user wishes to keep data private. BitTorrent doesn't offer the authorization framework to ensure privacy of caches. Users may mark the cache as only allowing DIRECT replications, which are encrypted and authenticated.

\subsection{Downloading Caches}
When a job starts, the CacheD begins to download the cache file.  The cache is identified by a unique string that includes the cache's name and the cache's originator host.  The flow of replication requests is illustrated in Figure \ref{fig:replicationflow}.  The replication requests originate from the file transfer plugin, which sends the replication request to the node local CacheD.  The node local CacheD then sends the replication to its parent or the origin cache.  The propagation of replication requests are modeled after well-known caching mechanisms such as DNS.

\begin{figure}[ht]
\centering
\includegraphics[width=0.5\textwidth]{images/CacheDownloadFlow.pdf}
\caption{Flow of Replication Requests}
\label{fig:replicationflow}
\end{figure}

\begin{enumerate}
\item The plugin contacts the node local CacheD daemon on the worker node.  It requests that the cache is replicated locally in order to perform a local transfer.
\item The node local CacheD responds to the file transfer plugin with a ``wait'' signal.  The file transfer plugin polls the node local CacheD periodically to check on the replication request.
\item The local CacheD daemon propagates the cache replication request to its parent, if it exists.  If the CacheD does not have a parent it contacts the cache originator in order to initiate a cache replication.
\item If the cache is detected to be transferable with BitTorrent, the download begins immediately after receiving the cache's metadata from the parent or origin.
\item Once the cache is replicated locally, the plugin downloads the files from the local CacheD.
\end{enumerate}






Each download is negotiated for the appropriate transfer method between the parent and the client.  Between parent and client CacheD's, the cache's individual replication preferences are honored.  Between a CacheD and the transfer plugin, an additional protocol is offered: symbolic link (symlink).

If the transfer plugin successfully authenticates with a local CacheD, transfer methods are negotiated.  If supported, the symlink method may be chosen.  The symlink transfer method allows near instant transfer of the cache from the CacheD to the plugin.  A symlink is created by the CacheD in the job's working directory pointing to the cache directory.  This symlink method eliminates transferring the cache to each job.

\begin{figure}[ht]
\centering
\includegraphics[width=\textwidth]{images/ReplicationBottleneck.pdf}
\caption{Cache Replication Showing Bottleneck}
\label{fig:cachebottleneck}
\end{figure}

In Figure \ref{fig:cachebottleneck}, you can see a traditional configuration of a cluster.  The configuration shows that there is a Network Address Translation bottleneck or a network bottleneck between the submit machine and the execution nodes.  The bottleneck limits the bandwidth between the submit machine and the execution nodes.

% Discuss the possiblity of jobs modifying the cache?


\subsection{Parenting of CacheDs}
During testing of the CacheD, it was apparent that BitTorrent increases the IO queue on the host server significantly, degrading the IO performance for all jobs on the server.  This increased IO queue leads to competition between BitTorrent-enabled CacheD's on the same host.  In order to address the increased IO queue, each CacheD will designate a single daemon on the host that downloads the files through BitTorrent.  All other CacheD�s will then download the cache from the parent using Direct file transfer mechanisms.  

\section{Results}

\subsection{Experimental Design}
To evaluate our solution, we will run a BLAST benchmark from UC Davis \cite{blastbenchmark}.  We chose a BLAST benchmark due to many factors.  BLAST is used frequently on campuses, but used infrequently on clusters due to the size of the database. BLAST has very large databases that are required by each job.  This makes it difficult to use on distributed resources since each job requires significant data.
BLAST databases are frequently updated, making them poor candidates for static caching, but good candidates for short-term caching, for which our CacheD specializes.

The BLAST database distributed with the benchmark is a subset of the Nucleotide NR database.  In our tests, we will use a larger subset of the NR database in order to demonstrate the efficiency of our solution.

For researchers, the time to results is likely the most important metric.  The stage-in time of data can be a large component of the entire workflow time.  We will measure the time for stage-ins as well as the average stage-in time.

We designed two experiments that represent our experience on campus infrastructure.  In the first experiment, we will allow 100 simultaneous jobs to start at the same time and measure the average download time versus the number of distinct nodes.  This experiment also includes the download time for child caches.  We chose 100 jobs somewhat arbitrarily in order to completely fill all of the nodes we were allocated on the cluster.  

In the second experiment, we compare the total stage-in time for a variable number of jobs while number of distinct nodes remains constant at 50.  This will show that the cache is working to eliminate transfer times when the files are already on the node.  Further, it will compare HTCondor's File Transfer method versus the CacheD's two transfer methods: BitTorrent and Direct.

When the number of jobs is fewer than 50, each job must download the cache since there are 50 nodes available for execution.  When the number of jobs is more than 50, all jobs that run after the initial download use a cached version of the data.

In our experiments, each job will use the CacheD to stage-in data to the worker nodes.  The jobs will be submitted with glideins created by Bosco \cite{weitzel2014accessing}  and the Campus Factory \cite{weitzel2011campus}.  Bosco allows for remote submission to campus resources while the Campus Factory allows for on-demand glidein overlay of remote resources.  The Campus Factory is used in order to create and run glideins which, in turn, run the CacheD daemon.  Bosco was used in order to submit to multiple campus resources simultaneously.

These two experiments were conducted on a production cluster at the Holland Computing Center at the University of Nebraska--Lincoln (UNL).

\subsection{Results}

We completed 41 runs of the BitTorrent versus Direct transfer  experiments on the UNL production cluster.  We first confirmed our suspicion that the Direct transfer method would result in a linear increase in the average stage-in time to transfer the cache as we increased the number of distinct nodes.  Conversely, we found that the BitTorrent transfer method did not significantly increase the average stage-in time as we increased the number of distinct nodes.  The BitTorrent transfer method was faster than the Direct in all experiments.


\begin{figure*}[h!t]
\centering
\includegraphics[width=\textwidth]{images/CombinedPlot.pdf}
\caption{Comparison of Direct and BitTorrent Transfer Methods with Increasing Distinct Node Counts}
\label{fig:combinedgraph}
\end{figure*}

Figure \ref{fig:combinedgraph} shows that the BitTorrent transfer method is superior to Direct for all experiments that were run.  Since multiple CacheDs on the same node will parent to a single CacheD, the number of distinct nodes is the dependent variable.  After the parent cache downloads the cache for the node, then each child cache will download from the parent using the Direct transfer method.

The Direct method of transfer follows a linearly increasing time to download the cache files.  This can be explained by bottlenecks of the transfers between the host machine and the execution nodes.  The increase in number of distinct nodes increases the stage-in time for any individual node.

The average download times for BitTorrent stage-ins are also shown in Figure \ref{fig:combinedgraph}.  The stage-in time does not significantly increase as the distinct nodes increases.  This meets our expectations.  We expect this trend to continue as the number of distinct nodes increases since BitTorrent can use peers to speed up download time.

\begin{figure}[ht!]
\centering
\includegraphics[width=0.5\textwidth]{images/modes_vs_downloadtimes.png}
\caption{Historgram of Transfers Mode vs Download Times}
\label{fig:histmethod}
\end{figure}

To better illustrate how parenting affects the download time of a cache, we show a histogram of the different modes in Figure \ref{fig:histmethod}.  The figure shows that while the parents download first, and nearly at all the same time, the children take a variable amount of time to download.  This variability can be attributed to the number of children on a node.  The more children downloading the cache at the same time, the slower each download will take. 

For our second experiment, we calculated the total stage-in time for a variable number of jobs.

\begin{figure*}[ht!]
\centering
\includegraphics[width=\textwidth]{images/StageinPlot.pdf}
\caption{Transfer Method vs Number of Jobs}
\label{fig:methodvsnumjobs}
\end{figure*}

When we limit the number of nodes to 50, we can clearly see the effect of the caching by varying the number of jobs.  In Figure \ref{fig:methodvsnumjobs}, both the Direct and BitTorrent transfer methods have a natural bend at about 50 jobs.  This correlates to when the CacheD has on-disk caches of the datasets, and the transfer to the job's sandbox is nearly instantaneous.  

The HTCondor file transfer method has a shorter stage-in time for low numbers of distinct nodes than the Direct method.  This can be explained by the increased overhead that the CacheD introduces when transferring datasets.  After all 50 nodes have the dataset cached locally, the Direct transfer method becomes more efficient than the HTCondor file transfers.

%it begins with less overhead than the direct or the BitTorrent methods at low job counts.  But as the number of jobs increase, so to does the total stage-in time.  Since the condor file transfer method does not cache the data, it continues to linearly increase in stage-in time as the number of jobs increase.



\section{Conclusions}
We have presented the HTCondor CacheD, a technique to decrease the stage-in time for large shared input datasets.  Our experiments proved that the CacheD decreases stage-in time for these datasets.  Additionally, the transfer method that the CacheD used can significantly affect the stage-in time of the jobs.

The BitTorrent transfer method proved to be a efficient method to transfer caches from the originator to the execution hosts.  In fact, the transfer time for jobs did not increase as the number of distinct nodes requesting the data increased.  Any bottlenecks that surround the cluster are therefore irrelevant using the BitTorrent transfer method.

%For caching methods that attempt to optimize per cluster access, such as HTTP proxy methods, the results would like be very similar to those shown above.  Per cluster caching still bottlenecks the transfers to a single or set of nodes near the cluster.  They are better for optimizing latency of small accesses rather than aggregate bandwidth, which is required for large input datasets.

In the future we plan to investigate incorporating job matchmaking with cache placement.  The HTCondor Negotiator could attempt to match jobs first against resources that have the input files before matching against any available computing resources.


\section*{Acknowledgment}

This research was done using resources provided by the Open Science Grid, which is supported by the National Science Foundation and the U.S. Department of Energy's Office of Science.





\section{Introduction}

Computation and storage are intrinsically linked.  Further, it is necessary for the computation to communicate its requirements and preferences to the storage.  In order to meet these demands, I have designed a new file transfer service that can be used to coordinate storage with the computation.  

Large input datasets are becoming common in scientific computing.  Unfortunately, on the Grid there is no solution for many use cases.  The typical use case that we see on the Grid is:

\begin{itemize}

\item User has a set of jobs, for example 1000
\item There are 200 machines available.
\item Each job has a large number of shared input files.

\end{itemize}

Since each machine will run multiple jobs, by default, all batch systems will transfer the large input file for each job.  Our goal is to minimize the number of times the input data is transferred from the submit host to the execution target.  As it was described in the previous chapters, a shared filesystem is frequently not available when submitting to multiple campus clusters, or on the Grid.  Therefore a framework of local caching and distributed file transfer is proposed to be developed to address these situations.

Local caching is defined as saving the input files on the local machine and making them available to local jobs.  Local caching is different from site caching, which is done in the OSG by squid caches.  We define site caching in which data files are stored and available to jobs from a closer source than the original.  In most cases on the OSG, the site cache is a node inside the cluster that has both low latency and high bandwidth connections to all of the execution hosts.

We use distributed transfer to mean transfers that are not from a single source.  In our case, we will be using Bittorrent \cite{cohen2008bittorrent}, which a client may download parts of files from multiple sources.  Additionally, the client may make available parts of the file that have already been downloaded.

When transferring files in a distributed framework, network bandwidth can be a limiting factor. 


\section{Requirements}

The requirements of the caching are:
\begin{itemize}
\item Minimize the number of times the input files are transferred.
\item Coordinate transfers with job submissions.
\item Provide a intuitive user interface for ease of use.
\item Each cached item must have a lease and an expiration to enable the deletion of items.
\end{itemize}



\section{Design}

The caching daemon is run on both the submit and execute hosts.  It acts as an independent agent managing the host's cache, while interacting with the both user and other deamons that may make requests to the cache.



\subsection{User Interaction}

A user, or a software agent on their behalf, will copy files into the cache and will assign it a lease with an explicit expiration time.  The user will provide a replication policy expression that will be used to determine to which execution hosts to replicate the files.

The caching daemon that is running on the submit host will manage the user's cache.  The user communicates with the caching daemon in order to copy files into the cache and to assign attributes to the cache, including expiration time and replication policy.  After the expiration time, the local daemon can choose to delete the files in the cache.

The user may assign a replication policy to the cache.  The replication policy has 2 components, the requirements on the nodes which to replicate and the method used to replicate the data to other caches.  The replication policy may include the methods that can be used to replicate and transfer the files.  For example, for private files, secured transfers are required.  While, for public data, you may be able to use bittorrent which in most situations will be no worse than direct transfers.  The cache originator has a default replication policy set by an administrator.

An important concept of the caching framework is a cache originator.  The cache originator is the original daemon that the user uploaded their input files.  A cache originator is in charge of distributing replication requests to potential nodes, as well as providing the cached files when requested.

\subsection{Daemon Interaction}

% Caching daemons register with the collector
Each caching daemon registers with the pool collector.  The collector serves as a catalog of available cache daemons that can be used for replication.  Additionally, each cache originator advertises their original caches to the pool collector.

The caching daemons interact with each other during replication requests.  A cache originator sends replication requests to remote caching daemons that match the replication policy that is set by the user.  The remote caching daemon then confirms that the cache data can be hosted on the server.  The remote cache then initiates a file transfer in order to transfer the cached data from the origin to the remote cached.

A pull model was designed in order to increase the autonomy of the caching daemons.  The cache originator daemon will send requests to the remote caching daemons.  But it does not expect a response.  It is up to the remote caching daemon whether to pull the files into it's own cache.  This allows the remote caching daemon the autonomy to deny a replication request for any reason.  Though the cache originator does attempt to match caches with remote resources based on the advertised resource requirements, the remote cache state may change between advertisements, and therefore deny a replication even if the originator matches the resource with a cache.  Reasons to deny a caching replication request could be:

\begin{itemize}
\item The resource does not have the space to accommodate the cache.
\item The resource may not have the necessary bandwidth available in order to transfer the cache files.
\item The resource does not expect to be able to run the user's jobs and has determined that replicating the cached files will not be used.
\end{itemize}

With the autonomy that the pull model, the caching daemons are able to refuse a replication request without causing faults in the system.

\subsection{Job Interaction}





The policy expression language is modeled after the matchmaking language in the Condor system \cite{raman1998matchmaking}.  The caching daemon is matching the cache contents to a set of resources, therefore it is natural to use Condor's matchmaking language that is used to match jobs to resources.  Once a resource is determined to match the caching contents policy expression, the caching daemon will contact the resource's caching daemon in order to initiate a cache replication.  The caching daemon on the remote resource is an independent agent that has the ability to deny a caching replication even after matchmaking is successful.  

In addition to traditional source and sink transfers, the caching daemon will employ a group transfer method modeled after the bittorrent \cite{cohen2008bittorrent} transfer mechanisms.  It will utilize the shared transfer techniques of Bittorrent in order to minimize the bandwidth between the execution resource and the submit host.  The execution host will download data not only from the submit host, but also from other nearby execution hosts.  Most resources will be on network partitions, therefore communication between clusters will likely be impossible.  Communication within the cluster is traditionally optimized, therefore execution resources talking with one another will be most efficient.

\section{Usage}
The command line usage of this caching daemon has not been fully designed.

\section{Tests}

There are many tests that can be used to determine if we meet our requirements.  I will list a few here.  As the implementation matures, further tests may be developed.

\begin{itemize}
\item Cache Hit Rate - The percentage of the time that jobs transfer input files from the cache rather than from the submit host.
\item Replicated Bytes - Number of bytes that are sent multiple times from the submit host to any execution host.
\item Time to completion - Does the caching improve the total time to completion for workflows?

\end{itemize}

Many of these tests are dependent on the workflow used to evaluate the caching.  Therefore, we will use sample workflows from the community such as a BLAST and AutoDock to evaluate the solution.

\subsection{Expected Results}

If the caching daemon implemention is successful, I anticipate that the cache hit rate is high and the replicated bytes sent from the submit host is low.  Further, with the Bittorrent transfer method, I would expect the bytes sent inside the cluster to be much larger then the bytes copied from the submit host, which is preferable since the bandwidth between nodes inside the same cluster is typically much larger than the bandwidth from an external source, such as the submit host.

\section{Conclusion}

Work on the daemon has begun.  A full design document has been create, which includes protocol specifics and prototyped usage of the caching framework.  Implementing the protocol handlers and the matchmaking is to be completed.





%% backmatter is needed at the end of the main body of your thesis to
%% set up page numbering correctly for the remainder of the thesis
\backmatter

%% Start the correct formatting for the appendices
\appendix

%% Appendices go here (if you have them)

%% Bibliography goes here (You better have one)
%% BibTeX is your friend

\bibliographystyle{plain}
\bibliography{DerekWeitzelDissertation}

%% Index go here (if you have one)
\end{document}

\endinput
%%
%% End of file `skeleton.tex'.
