%%
%% This is file `skeleton.tex',
%% generated with the docstrip utility.
%%
%% The original source files were:
%%
%% nuthesis.dtx  (with options: `skeleton')
%% 

%%
%% For common degrees, you can use the class options:
%% phd, edd, ms, ma
%% phd is the default
\documentclass[print,phd]{nuthesis}

\usepackage{caption}
\usepackage{framed}
\usepackage{graphicx}
\usepackage{listings}
\usepackage{hyperref}
\usepackage{bibentry}
\usepackage{amsmath}
\usepackage{algpseudocode}
\usepackage{algorithm}
\usepackage{wrapfig}
\usepackage{lscape}
\usepackage{rotating}
\usepackage{epstopdf}
\usepackage{color}
\usepackage{subfig}
\usepackage{afterpage}
\nobibliography*

\lstset{
	breaklines=true, 
	breakatwhitespace=true,
	frame=lines,
	basicstyle=\ttfamily,
	showstringspaces=false
}

\definecolor{grey}{rgb}{0.9,0.9,0.9}

\setcounter{tocdepth}{1}

\begin{document}
%% Start formatting the first few special pages
%% frontmatter is needed to set the page numbering correctly
\frontmatter

\title{Enabling Distributed Scientific Computing on the Campus}
\author{Derek Weitzel}
\adviser{Professor David Swanson}
\adviserAbstract{David Swanson}
\major{Computer Science}
\degreemonth{July}
\degreeyear{2015}
%%
%% For most people the defaults will be correct, so they are commented
%% out. To manually set these, just uncomment and make the needed
%% changes.
%% \college{Your college}
%% \city{Your City}
%%
%% For most people the following can be changed with a class
%% option. To manually set these, just uncomment the following and
%% make the needed changes.
%% \doctype{Thesis or Dissertation}
%% \degree{Your degree}
%% \degreeabbreviation{Your degree abbr.}
%%
%% Now that we know everything we need, we can generate the title page
%% itself.
%%
\maketitle
%%
%% You have a maximum of 350 words for your abstract, which includes
%% your title, name, etc.
%%
%% Required
\begin{abstract}
Campus research computing has evolved from many small decentralized resources, such as individual desktops, to fewer, larger centralized resources, such as clusters.  This change has been necessitated by the increasing size of researcher's workloads, but this change has harmed the researcher's user experience.  We propose to improve the user experience on the computational resources by creating an overlay cluster they are able to control.  This overlay should transparently scale to national cyberinfrastructure as the user's demands increase.

We explore methods for improving the user experience when submitting jobs on a campus grid.  To this end, we created a remote submission and overlay computational framework called Bosco.  This framework can remotely submit processing from the user's laptop to clusters on the campus or on national cyberinfrastructure.  To illustrate the possibilities of improving the user experience of remote submission, we created BoscoR, an interface to Bosco in the popular statistics and data processing programming language, R.  Bosco improves the user experience of submitting to campus clusters, while also being an efficient method for job management.

In order to solve some of the issues with data distribution on opportunistic resources, we created the CacheD, a data management framework for managing and provisioning storage resources on the campus.  The CacheD additionally optimizes transfers to multiple resources by using the peer-to-peer transfer protocol, BitTorrent.  Further, the CacheD optimizes shared data between multiple jobs by caching the input data directly on the execution resources.  The CacheD decreases the stage-in time over current transfer methods and significantly decreases stage-in time when the data is already cached.

Finally, we explain how to control data distribution on a campus through a comprehensive policy framework.  This framework is implemented in the CacheD.  We present the policy language, available attributes, and how to extend the policy language beyond the default behavior.  Multiple examples are given for different data distribution scenarios observed on campus resources.   

Combining easy-to-use campus job submission with Bosco, efficient data distribution with the CacheD, and a policy language to manage the data distribution, we have created a unified framework for campus computing.
	
	
	
%Campus research computing has evolved from many small decentralized resources, such as individual desktops, to fewer, larger centralized resources, such as clusters.  This change has been necessitated by the increasing size of researcher's workloads, but this change has harmed the researcher's user experience.  When the resources were decentralized, they may have been under the user's desk or just down the hall from their research lab.  They had privileges to install software that could control the scheduling of their computationally intensive jobs and manage their storage resources.  Now, as we have moved towards fewer but larger centralized computational resources, the users have fewer privileges than they enjoyed previously.  They cannot install software on the centralized campus clusters.  The scheduling of their jobs and the storage is managed centrally.
%
%We propose that users should have their privileges restored onto the computational resources by creating an overlay cluster they are able to control.  Both job submission and data distribution should come from the user's computer to the campus resources.  Additionally, these methods should transparently scale to national cyberinfrastructure as the user's demands increase.
%
%First, we explore methods for improving the user experience when submitting jobs on a campus grid.  To this end, we created a remote submission and overlay computational framework called Bosco.  This framework can remotely submit processing from the user's laptop to clusters on the campus or on national cyberinfrastructure.  To illustrate the possibilities of improving the user experience of remote submission, we created BoscoR, an interface to Bosco in the popular statistics and data processing programming language, R.  Bosco improves the user experience of submitting to campus clusters, while also being an efficient method for job management.
%
%Second, we discuss the issues with data distribution on the campus.  In order to solve some of the issues with data distribution on opportunistic resources, we created the CacheD, a data management framework for managing and provisioning storage resources on the campus.  The CacheD additionally optimizes transfers to multiple resources by using the peer-to-peer transfer protocol, BitTorrent.  Further, the CacheD optimizes shared data between multiple jobs by caching the input data directly on the execution resources.  The CacheD decreases the stage-in time over current transfer methods and significantly decreases stage-in time when the data is already cached.
%
%Third, we explain how to control data distribution on a campus through a comprehensive policy framework.  This framework is implemented in the CacheD.  We present the policy language, available attributes, and how to extend the policy language beyond the default behavior.  Multiple examples are given for different data distribution scenarios observed on campus resources.   
%
%Combining easy-to-use campus job submission with Bosco, efficient data distribution with the CacheD, and a policy language to manage the data distribution, we have created a unified framework for campus computing.


\end{abstract}

%% Optional
%% \begin{copyrightpage}
%% \end{copyrightpage}

%% Optional
%% \begin{dedication}
%% \end{dedication}

%% Optional
\begin{acknowledgments}
%TODO: Make an acknowledgments
I am thankful to have worked with many fine people throughout graduate school.

Dr. David Swanson, my dissertation advisor, has provided a fertile environment of both distributed computing users and the resources to test and provide solutions.  In addition, I am forever thankful for the opportunity he provided for me to attend graduate school.

Dr. Brian Bockelman has provided immense guidance for my dissertation.  I thank Brian for mentoring me throughout my collegiate career.

I thank my colleagues at the Holland Computing Center.  I have broken their systems, asked for advice, and caused an untold amount of extra work for them.  I could not have completed this dissertation without the skilled experts at HCC who have helped me throughout the process.

I would also like to thank my colleagues in the Open Science Grid.  They have provided me with a fellowship and internships in support of my Ph.D.  I would like to especially thank Ruth Pordes and Dan Fraser for providing valuable advice and mentorship in my academic career.

My family deserves thanks for many intangible gifts.  I thank them for encouraging me to complete my goals.

Last but not least, I thank my lovely fianc\'ee Katie for providing immeasurable support. 

\end{acknowledgments}

%% Optional
%% \begin{grantinfo}
%% \end{grantinfo}
%% The ToC is required
%% Uncomment these if need be

%% The ToC is required
\tableofcontents
%% Uncomment these if need be
\listoffigures
\listoftables
%%
%% ``Real'' beginning of the document.
%% mainmatter is needed to set the page numbering correctly
%%   mainmatter is needed after the ToC, (LoF, and LoT) to set the
%%   page numbering correctly for the main body
\mainmatter

%% Thesis goes here

\chapter{Introduction}

In this dissertation, we optimize distributed computing workflows on a campus grid.  We are interested in optimizing a researcher's use of the computational and storage resources on the campus to increase the reliability and decrease the time to solution for scientific results.  We first extend prior work to enhance the computational capabilities of researchers on a campus.  We then expand our work to the data needs of modern workflows.

\section{Campus Grid Computing}

The increase of performance of computer hardware following Moore's law \cite{schaller1997moore} has allowed scientists to tackle larger problems.  As they increase their use of research computing, their applications far exceed the locally available resources.  Such applications often turn to distributed computing to aggregate more computational, memory, or storage resources than locally available resources can provide.

%Despite the ever increasing performance of computer hardware following Moore's law \cite{schaller1997moore}, applications continue to keep pace with hardware's capabilities as researchers tackle larger problems.  For some users, their applications far exceed the capabilities of computers that are immediately available to them.  Such applications may be able to use multiple computers to aggregate  more computational, memory, or storage resources than a single computer can \mbox{provide}.

Batch computing can combine the computational, memory, and storage resources of multiple computers in a single cluster through concurrent scheduling of applications.  A computational grid is an extension of batch computing, where resources may be combined from multiple pools of resources to be used for an application.

A computational grid is a hardware and software infrastructure that provides dependable, consistent, pervasive, and inexpensive access to high-end computational capabilities \cite{foster2004grid}.  A campus grid is a specialized grid where multiple resources are owned by the same organization, although it may be in multiple administrative domains.  
%For our discussion of computation, we restrict our consideration to those campuses that have multiple computational resources.

A campus grid has become necessary to spread demand across multiple clusters.  This is important when demand for a single cluster is large, due to improved performance or increased storage, and demand is low on other available clusters.  One aim for a campus grid is to move computation from the in-demand cluster to other clusters, which can result in a shorter time to completion for the users' jobs.

To succeed, a campus grid requires a framework to distribute jobs to multiple clusters in a campus.  In \cite{weitzel2011campus}, I proposed a solution based on HTCondor \cite{litzkow1988condor}.  The solution required installation of a campus factory \cite{website:campusfactory} on each cluster's login node.  An on-demand overlay was created that could efficiently run high throughput jobs on multiple campus resources.  

Although my solution was efficient and fault tolerant, it was deficient in several ways.  Installation and setup of the campus factory was difficult since it was not automated.  The communication inside the overlay was insecure.  We set out to correct these deficiencies.

%Users had to install HTCondor on both the cluster's login node and the user's submit node.  Also, the security setup was based on IP whitelists, which can be defeated with IP spoofing.  Therefore, we set out to correct these deficiencies.

We have enhanced my Masters thesis' solution to include:
\begin{itemize}
\item Easier installation through automation
\item Increased security through secure key exchange
\item More supported cluster types and configurations
\item Improved access to computing through language frameworks such as R \cite{team2005r}
\end{itemize}

We created a framework for job submission to remote resources that the user does not control.  Typical grid submission uses custom interfaces such as the Globus Resource Allocation Manager (GRAM) \cite{foster1999globus}, which is previously installed by an administrator.  We assume resources do not have such dedicated grid software installed.  This framework does not require administrator intervention for remote submission to opportunistic resources.  The framework uses interfaces that are installed on nearly all clusters that are typically used for interactive access.  It automates the submission and error handling of jobs submitted to remote resources, while providing the user a consistent interface over multiple, load-balanced clusters.

The new framework is named Bosco \cite{weitzel2014accessing}.  It uses secure protocols to connect to remote clusters in order to transfer files and submit/monitor jobs.  Installation of Bosco on remote clusters and the submit host has been automated with simple tools.  Clusters with restrictive firewalls are supported by multiplexing operations through a single secure connection.  Furthermore, many cluster schedulers are supported by the underlying technology.  A diagram of the architecture of Bosco is shown in Figure \ref{fig:introboscoarch}.

\begin{figure}[h!t]
	\centering
	\includegraphics[width=\textwidth]{images/ArchitectureGraph1.pdf}
	\caption{Bosco Architecture}
	\label{fig:introboscoarch}
\end{figure}

Bosco, in coordination with technologies in HTCondor, enable a job distribution method which is provisioned based on demand.  A default Bosco installation is able to submit to one local cluster.  If that cluster does not meet the user's computational needs, then Bosco can be configured to submit to multiple clusters with load balancing between them.  If the user's computational needs are still not met with multiple clusters, they can configure Bosco to submit resource requests to national cyberinfrastructure such as the Open Science Grid (OSG) \cite{pordes2007open}.  The provisioning capabilities of Bosco creates an ever expanding network of available resources. The goal is to provide an expanding network of resources as shown in Figure \ref{fig:boscogrowing}.

\begin{figure}[h!t]
	\centering
	\includegraphics{images/BoscoGrowing.pdf}
	\caption{Bosco's Growing Reach as Demand Increases}
	\label{fig:boscogrowing}
\end{figure}

In order to ease access to Bosco for data processing, an interface has been developed in the most widely used data processing language, R.  This BoscoR framework enables users to never leave their R environment in order start remote data processing.

But Bosco is not enough for researchers that have large data requirements.  Input and output data are explicitly listed by the user.  The data is transferred over the secure, but slow, connection between the submitter and resource for every job.  Therefore, we must consider data and storage management on the campus grid.


%Most major research campuses, whether a university campus, or a national lab campus, have a research computing resource.  The computing resources are broken into two categories:

%\begin{itemize}

%\item Condominium - Resources are purchased by research groups for their dedicated use.  They are added to a cluster that may share infrastructure such as a filesystem or an interconnect.
%\item Shared resources -  Resources are purchased by a central authority that are shared between multiple research groups.

%\end{itemize}





\section{Data Management on Campus}

There are many challenges in data management and distribution in scientific computing \cite{deelman2008data}.  For batch computing, one challenge is transferring the data from the user's computer to the execution resources.  Large data workflows can strain the network near the data's source, which can result in unreasonable amounts of batch time used solely for data transfer.

Data management is the framework and policies controlling data through the research cycle.  In this dissertation, we are concerned with optimizing data management when using campus computational resources.

As users spread their computation across multiple clusters either on the campus or across campuses, data distribution and collection becomes more difficult.  Before using the campus grid, a user would select a cluster to do their processing.  The user then could host all of their data on that cluster by copying the data onto that cluster's shared filesystem.  The jobs access the data from the shared filesystem just as it would on the user's desktop, available for all executions at the same directory.

These assumptions do not hold for a campus grid.  A grid is made up of multiple computational clusters, with potentially many separate filesystems; no single filesystem is accessible access from every computational resource.  Further, the shared filesystem could become a bottleneck if many jobs are requesting the same data simultaneously.  Therefore, data management techniques must evolve along with computation.  


Most distributed batch schedulers are able to transfer the input data for each job execution.  Each job starts with an empty execution area and the scheduler will transfer the files into it.  When the user is not using the scheduler to transfer data, the input data must still reach the execution host.  Data will be transferred from the source (usually the user's computer) to the execution resources for processing.  The network connection between the source and the execution resources may be a bottleneck for the computation.  Frequent re-transfers of the same input data will further congest the network between the source and the execution resources.

In this dissertation, we optimized two attributes of distributed data management: efficient transfer methods and reduction of duplicate transfers.

We introduce the CacheD \cite{weitzel2015pdpta}, a caching and data transfer daemon for input data in distributed computing.  The CacheD uses novel data management methods based on technology developed for large peer-to-peer data transfers on the Internet, BitTorrent.  It also caches input data on the execution resources to enable quick transfers on subsequent requests for the same input data.

Similar to the work with Bosco, the CacheD does not require privileged access in order to provision storage resources.  It can use the storage on worker nodes spread across multiple clusters as a data input caching system.

\section{Data Distribution Policy Language}

Users of grid submission software currently have to describe how their files will be transferred from their submission host to the remote execution resource where the data will be processed.  They have to coordinate the storage and computational resources without help.  We propose a policy language that allows an agent to decide an appropriate method for data transfer.  It determines the transfer method by negotiating between the following three sources: a user-given policy language for the data, the remote execution resource's capabilities and preferences, and the submitting resource's capabilities and preferences.  In addition, the policy language should determine if the cache should be replicated to multiple resources.  A modern flexible policy language for describing data distribution for campus users is needed.

In addition to the CacheD described above, a policy language must be made in order to help the CacheD make decisions when interacting with other agents, such as other CacheDs or the local node.  We discuss extending this policy language to include custom attributes that users can include to improve choices on data distribution.

The policy language utilizes the \mbox{ClassAds} \cite{raman1998matchmaking} language.  These \mbox{ClassAds} were originally developed in the context of matchmaking between computational resources and potential jobs.  ClassAds are a schema-free language for describing heterogeneous resources.  We demonstrate usage with new attributes that pertain to storage and expressions that can be evaluated to make decisions.

The user must specify preferences for the cache to consider.  Examples include: where should this cache be distributed, how should the cache be distributed, and how long the cache should be stored.  Each of these preferences must be negotiated with the preferences of the CacheD that may store or is storing the cache.  The user's preferences will affect how fast the cache is transferred (different transfer methods are more efficient than others) and also, on which and how many nodes that cache should be replicated.

Further, each CacheD must coordinate with one another in order to distribute the caches in an efficient method.  Replication of caches between CacheDs must be negotiated.  A CacheD may decide, through evaluating its own policies, whether or not to accept a cache to be stored.  These policies are again expressed in the ClassAd policy language.



\section{Overview of Dissertation}

This dissertation describes how data intensive applications can be run in a distributed campus environment.

\begin{description}
	\item[Chapter \ref{chapter:relatedwork}:]  There are many distributed computing platforms available publicly.  In this chapter, we will discuss these schedulers and differentiate them with Bosco.  Also, we will discuss other available data management, distributed storage, and caching systems.
	
	\item[Chapter \ref{chapter:campusjobs}:] We will discuss how computing can be managed on the campus using the Bosco framework.  We will also discuss a case study of integrating Bosco with the programming language, R, in order to provide an easy-to-use interface to campus distributed computing.
	
	\item[Chapter \ref{chapter:campusdatadistribution}:] We will discuss the CacheD, a campus data distribution service.  The CacheD is able to combine novel transfer methods with data caching to improve the stage-in time for large data sets.  Through evaluation, we show that the CacheD demonstrates a significantly shorter stage-in time for large data sets over existing solutions that have been deployed.
	
	\item[Chapter \ref{chapter:campusstoragepolicylanguage}:]  
	Simply caching and transferring data does not provide the flexibility that the CacheD requires to operate in a distributed environment.  In this chapter, we will discuss the policy framework and language that enable the CacheD to interact with the user and other daemons in order to make decisions.
\end{description}


\newpage
\section{A Note on Terms}
High performance and distributed computing often use terms inconsistently.  Below is a definition list of such terms, and how we will define them for this dissertation:


\begin{description}
	\item[Job:] A packaged unit of work with input and output.  A job may consume computational, memory, network, and/or storage resources in a batch system.
	\item[Workflow:] A logical grouping of jobs executed on resources.  The jobs may have some ordering.
	\item[Campus:] An organization that may have multiple administrative domains which may vary access policies to resources.
	\item[Execution Resource:] A resource which fulfills the requirements of a job and may also run it.  This may be a worker node in a cluster.
	\item[Cluster:] A set of execution resources that have high interconnection bandwidth and are managed by a single scheduler.
	\item[Batch System:] A scheduler for the resources of a cluster.
	\item[Agent:] An independent entity that can make decisions on its own without the control of another entity.  In this dissertation, we will use the word agent to describe a daemon which can make independent decisions without the explicit control of other daemons.
	\item[Pilot:] Pilot jobs are containers that once started, will request work from the user's job queue.
\end{description}







\chapter{Related Work}
\label{chapter:relatedwork}

\section{Distributed Batch Systems}


Several batch systems and grid schedulers are able to schedule tasks on execution resources.  Examples of cluster schedulers that are frequently used are PBS \cite{pbstorque}, \mbox{HTCondor} \cite{litzkow1988condor}, and Slurm \cite{yoo2003slurm}.  Each scheduler is very good at resource management within a single administrative domain.  Each of these resource managers has a very limited ability to send processing to remote resources, which are typically under a separate administrative domain.  PBS and Slurm can send jobs between clusters that run the same schedulers.  HTCondor also has the ability to send processing to other clusters running HTCondor, and it can also transform jobs to the language of other schedulers such as PBS and Slurm.

Grid schedulers have become more popular as the number of resources has increased.  Examples of grid schedulers are OSGMM \cite{website:osgmm} and GlideinWMS \cite{sfiligoi2008glideinwms}.  These schedulers are able to send jobs to remote resources using grid protocols.  OSGMM performs a direct grid submission to the remote resources using the GRAM  \cite{foster1999globus} interface.  GlideinWMS also submits to the GRAM interface of the cluster, but provides an overlay of HTCondor daemons on top of remote resources.  The overlay presents a consistent HTCondor interface to the computing resources for ease of use.  


\section{Distributed Storage Access}

Distributed file systems have long had their own storage access methods.  An example of this is Hadoop \cite{white2012hadoop}, a popular distributed file and processing system.  The only method to access Hadoop storage is through the Hadoop protocol.  On the Open Science Grid, the primary access methods are through file system independent middleware such as the Storage Resource Manager (SRM) \cite{shoshani2002storage} and XRootd \cite{dorigo2005xrootd}.  They provide a translation layer from system independent grid protocols and security mechanisms to the underlying storage system, such as Hadoop.  The Storage Resource Manager (SRM) is a previously popular protocol to access remote distributed filesystems.  It is a standardized protocol that allows remote, distributed access to large storage with APIs to balance transfers among many data servers.


Beyond storage access methods are storage schedulers.  These schedulers do not define a protocol to access the storage, rather they coordinate the access.  NeST \cite{bent2002flexibility} is a software-only grid aware storage scheduler.  It supports multiple transfer protocols into a storage device, including GridFTP \cite{allcock2005globus} and NFS \cite{walsh1985overview}.  Further, it provides features such as resource discovery, storage guarantees, quality of service, and user authentication.  It is layered over a distributed filesystem to provide access to it.  NeST functions as the interface and access scheduler for a storage device.  Features such as the storage guarantees and quality of service require NeST to be the only interface into the storage device, a very rare feature in today's grid storage.  Today's storage elements, such as the 3 petabyte storage at University of Nebraska, include multiple interfaces to access the storage element.  Nebraska runs at least four methods of accessing and modifying storage \cite{attebury2009hadoop}, SRM \cite{shoshani2002storage}, GridFTP, XrootD, and Fuse \cite{szeredi2010fuse} mounted Hadoop.  All of these methods are required for compatibility with different access patterns and clients.  NeST could implement each of these protocols, but it would be extremely difficult to manage the storage centrally.  For example, Fuse is mounted on all 300 worker nodes.  The GridFTP and XrootD servers run on 10s of servers, with an aggregate bandwidth of 10 Gbps.  Scaling quality of service and storage allocation/enforcement across all of these access methods would likely prove impossible.



\section{Data Transfer Mechanisms}

A popular consumer transfer method, BitTorrent, has been used for data transfer in computational grids by Wei, Fedak, \& Capello \cite{wei2005collaborative, wei2005scheduling, wei2007towards}.  It has been shown to improve data transfer speeds when compared to traditional source and sink transfer methods, in this case FTP \cite{postel1985file}, which is very similar in architecture to GridFTP, a grid enabled FTP protocol.  The researchers did not compare performance of the BitTorrent protocol when compared to modern grid transfer techniques, such as using HTTP caching.  Further, the authors did not test BitTorrent transfers across network partitions that are common on the grid.  For example, a worker node from one cluster may not be able to communicate directly with another cluster.  Therefore, BitTorrent may not work between clusters but will work inside clusters.

Globus Online \cite{foster2011globus} is a web interface for transferring files between sites and sharing data with other users.  It offers an intuitive web interface for bulk transfers between endpoints.  It only supports the GridFTP \cite{allcock2005globus} transfer protocol and requires GridFTP implementations at all endpoints.

There are also popular data transfer tools used on clusters such as secure copy (SCP) from OpenSSH \cite{openssh} and rsync \cite{rsynce}.  SCP is a simple copy tool that uses the Secure Shell (SSH) protocol to transfer files from a source to a client.  Rsync is also able to copy files from a source to a client, but it can also do differential copies, where only the changed portion of a directory will be copied at a time.  Both of these methods are used heavily when the data is small.  But they both use single stream TCP in order to transfer data, which has been shown to be slower than multi-stream TCP which is used in GridFTP \cite{allcock2005globus} or BitTorrent.

\section{Data Management}

There have been previous policy frameworks for distributed storage.  These frameworks have largely been designed to move data between a few large filesystems.  Therefore, the interactions are rare but could make significant changes to the system.  In contrast, the CacheDs have frequent interactions with other agents, but each one has a minimal impact on the entire system.

Data management is different from data transfer and access in that it provides services on top of the file systems, such as meta-data storage and search capabilities.  An example of a data management service is the integrated Rule-Oriented Data System iRODS \cite{rajasekar2010irods}.  It provides metadata storage, querying, and rule-based placements.  Also, it can handle transfers to storage resources.  When given input, iRODS also has the capability to create rules and take actions on data.  It creates a small policy framework that upon certain actions, can execute micro-services.  This iRODS policy framework is much more extensive than what we created in Chapter \ref{chapter:campusstoragepolicylanguage}.  Our framework is designed for frequent interaction between many agents acting independently.  The rules for iRODS can be large and cumbersome for simple data replication.  Further, to do anything substantial with the rules, custom code must be written.

Stork \cite{kosar2004stork} is a data placement scheduler.  It can schedule data placement and transfers to and from remote storage systems.  Stork is innovative in that it treats data transfers similar to jobs.  It will queue transfers and check for proper completion of the transfers.

Kangaroo \cite{thain2001kangaroo} is another storage scheduler multi-level file access system.  It allows for multiple levels of staging in order to send job output back to a storage device.  It can do this by asynchronously staging data through multiple storage devices on its path to the destination filesystem.  The Kangaroo system only addresses output data.

DQ2 \cite{branco2008managing} and Phedex \cite{rehn2006phedex} are production transfer services for the Atlas and CMS physics experiments, respectively.  They are used to manage distributed transfers to and from sites inside the collaborations.  Additionally, they have had databases built on top of them that provide features such as combining files into datasets for easier bulk transfer management.  Both were designed for their experiments, and therefore, would be very difficult to generalize for outside users.

Distributed filesystems such as Hadoop also provide a small amount of data placement policy that can be configured.  For example, Hadoop can be configured to replicate the contents of a directory at least $X$ times.  Further, a script can be given to Hadoop which it can query to create a topology of the data center, further providing control of how the data replicas are sent.  This topology script has been used by me to create a data center aware Hadoop replication policy \cite{he2012hog}.

\begin{figure}[ht!]
	\centering
	\includegraphics[width=\textwidth]{images/BackgroundStorageDiagram2.pdf}
	\caption{Background on Storage Technologies}
	\label{fig:backgroundstorage}
\end{figure}

Figure \ref{fig:backgroundstorage} shows an overview of the different protocols and services, and how they fit into the three categories: Data Management, Data Services, and Transfer Protocols.

\section{State of Practice in Campus Computing}

In order to illustrate the available technologies on the grid, we will begin with a typical use case.  We will then describe the technologies that could enable this computing on the grid.

\subsection{Use Case}
We will begin with a typical use case.  In this particular case, we will focus on the use of BLAST \cite{altschul1990basic}.  BLAST workflows typically include the following files:

\begin{itemize}
	\item Executable
	\item Database
	\item Query Files
\end{itemize}

Each of these files have different properties.  The executable is relatively small, maybe 10s of megabytes.  But it is shared between all executions of BLAST.  The query files are unique to each job but are typically very small, not exceeding 1 MB.  

The database is a large collection of proteins that is searched for each protein in the query file.  Many databases are publicly available for use in BLAST.  The most common is the non-redundant (NR) database, which is currently 50 GB and updated weekly.

\subsection{Current Approach}

If the user has a BLAST application and wishes to run numerous jobs, they must first gain access to computational resources.  They may have access to a campus cluster.  In that case, they will log into the campus cluster.  The first step for the user is to learn the scheduler language.  There are many different languages, such as PBS, Slurm, or LSF \cite{computinglsf}.  All of these languages have slightly different syntax.

Once the scheduler language has been learned enough to write a submission file, the data must be transferred to the cluster from their laptop.  This is usually accomplished with a tool such as SCP from the OpenSSH \cite{openssh} package.  This will be transferred slowly as SCP only uses a single encrypted stream to send data.  For the 50 GB NR database from their wireless connected laptop to the cluster could take two hours (assuming 54 Mbps wireless), if nothing goes wrong with the transfer.

Once the data is on the cluster, the user will submit the jobs to the scheduler to process the data.  The BLAST database will be copied for each and every execution to the execution resources from the cluster's shared filesystem.

Once the computation has completed, the user will copy the output data back to their laptop for further analysis.

\subsection{Issues with Current Approach}

There are many issues with the current approach that I will point out.

\begin{itemize}
	\item Users must learn one or more scheduler languages.  If users wants to submit to only one cluster, then they only need to learn one submission language.  But if their demands grow, and they need more resources, they will need to learn another programming language.
	\item Data copies are very expensive and should be minimized.  The NR database is updated frequently;  therefore, it must be updated on the cluster frequently.
	\item Once on the cluster, each and every worker node will need to copy the NR database in order to process it.  This copy will happen every time, as there is no caching in the vast majority of distributed filesystems.
\end{itemize}









\chapter{Campus Job Distribution}
\section{Introduction}

In this chapter I will discuss the methods developed to aid in distributed scientific computing on a research campus.  

\section{Access to Computation on the Campus}

This could be merged with the introduction

Discuss
\begin{itemize}
\item Typical Campus resources
\item How users get access to these resources
\end{itemize}

\section{Bosco}

\begin{itemize}
\item How Bosco addresses the concerns in the above.
\item Why users would want to use Bosco over other solutions
\item Architecture
\end{itemize}

\section{Load Balanced Access to Computational Resources}

\begin{itemize}
\item Advantages of using the Campus Factory
\item Architecture
\item How the campus factory + bosco create utopia!
\item How the submission works using bosco
\end{itemize}





\section{Conclusion}







\chapter{Campus Data Distribution}

\label{chapter:campusdatadistribution}


% As a general rule, do not put math, special symbols or citations
% in the abstract
%\begin{abstract}

%A batch processing job in a distributed system has three clear steps, stage-in, execution, and stage-out.  As data sizes have increased, the stage-in  time has also increased.  In order to optimize stage-in time for shared inputs, we propose the CacheD, a caching mechanism for high throughput computing.  Along with caching on worker nodes for rapid transfers, we also introduce a novel transfer method to distribute shared caches to multiple worker nodes utilizing BitTorrent.  We show that our caching method significantly improves workflow completion times by minimizing stage-in time while being non-intrusive to the computational resources, allowing for opportunistic resources to utilize this caching method.


%\end{abstract}

This chapter is a combination of the following publication and additional work that is being prepared for publication.

\bibentry{weitzel2015pdpta}




\section{Introduction}

Large input datasets are becoming common in scientific computing.  Unfortunately for campus researchers, the staging time of the datasets to computational resources has not kept pace with the increase in dataset sizes.  The typical large dataset workflow may consist of thousands of individual jobs, each sharing input files.  

The campus resources made available to researchers are shared; therefore, the researchers have the limitation of not having access to install programs on the clusters.  My previous work, Bosco \cite{weitzel2014accessing}, built an overlay on top of campus resources to create a virtual, on-demand pool of resources for task execution.  I expanded the capabilities of this virtual pool to include data caching and novel transfer methods to enable big data processing.

%Each node in the virtual pool will run multiple jobs, by default, all batch systems will transfer the large input file for each job.  Our goal is to minimize the number of times the input data is transferred from the submit host to the execution target.  Therefore a framework of local caching and distributed file transfer is proposed to address these situations.

An excellent example of a big data workflow is that of the bioinformatics application: BLAST \cite{altschul1997gapped}.  Each BLAST query requires an entire reference database, which can range in size from a few kilobytes to many gigabytes.  The workflow to run a BLAST query requires a large stage-in time in order to make the reference database available.  Additionally, the databases are frequently updated with new entries.

Users in the past have copied the database using various methods.  The na\"{i}ve method includes copying the database for each job.  Storing the database on a shared filesystem has the same effect as copying the database for each job, since the database must be transferred to the execution node for each job.  I propose caching the database on the node for subsequent executions.

I find that the BLAST workflow described above is common among large data researchers.   

Bosco \cite{weitzel2014accessing} is a remote submission tool that can create overlay virtual pools designed for campus resources.  In previous work, Bosco allowed campus researchers to submit high throughput jobs to high performance clusters.  I extended Bosco to include data caching and novel data transfer methods. 

I limit the design and analysis to a campus cluster computing environment.  My solution is unique in that it is designed to run opportunistically on the campus computing resources.  Additionally, it does not require administrator intervention in order to create a virtual, on-demand pool of resources.


\section{Background and Related Work}

% Background on caching

Data caching on distributed systems has been used many times and at many levels.  Caching can be done on the storage systems and on the execution hosts, as well as in within the infrastructure separating the two.

Some distributed filesystems use local caches on the worker nodes.  GPFS \cite{schmuck2002gpfs} has a read-only cache on each worker node that can cache frequently accessed files.  It is designed for a fast, shared filesystem and is recommended when file access latency is a concern.  It is not recommended for large files since internal bandwidth to the local disk is assumed to be less than the bandwidth available to the GPFS shared filesystem.  GPFS file transfers are typically done over high speed interconnects which can provide high bandwidth for large files.  These interconnects are not typically available to users' jobs for transferring data from a remote source.

% pcache?

% HTTP caching
HTTP caching is used throughout the web to decrease latency for page loads and to distribute requests among servers.  In high throughput computing, a forward proxy is commonly used to cache frequent requests to external servers.  The forward proxy caches files requested through it, and it will respond to subsequent requests for the same file by reading it from memory or its own disk cache.

The HTTP forward proxy caching does have limitations.  The HTTP protocol was designed and is used primarily for websites.  Websites have very different requirements from high throughput computing.  The data sizes are much smaller.  Software designed as forward proxies, such as Squid \cite{squidcacheurl}, are optimized for web HTTP traffic, and therefore, do not handle large data file sizes optimally.  Further, the Open Science Grid (OSG) \cite{pordes2007open} sites typically only have one or possibly a few squid caches available to user jobs.  They are not designed to scale to large transfers for hundreds of jobs, the target use case.

% chirp / parrot
Parrot \cite{thain2005parrot} is another application that will cache remote files when using certain protocols.  Parrot uses interposition \cite{thain2001multiple} to capture and interpret IO operations by an unmodified binary application.  The interposition allows Parrot to provide a transparent interface to remote data sources.  Parrot caches some of those sources such as HTTP with GROW-FS, a filesystem using HTTP.  Parrot caches an entire file to the local storage.  Parrot must download directly from the source the first time it is requested, exhausting WAN bandwidth quickly for large files.


% CernVM-FS
CernVM-FS \cite{blomer2011cernvm} provides a filesystem over the HTTP protocol.  It integrates into the worker node system using the FUSE \cite{szeredi2010fuse} interface.  The CernVM-FS local node client caches files on the node, as well as using Squid to cache files at the site.  Again, since it uses the HTTP, it's not designed to cache large files.  Neither the Squid caches nor the web servers optimally transfer large files, nor are they designed for large data sizes.  Further, CernVM-FS requires administrator access in order to install and configure, a privilege that campus users do not have.


% xrootd caching
XrootD \cite{dorigo2005xrootd} is designed for large data access, and it has even been used for WAN data transfers \cite{bauerdick2012using} using a federated data access topology.  There has been some work in creating a caching proxy for the XrootD \cite{bauerdick2014xrootd}.  The caching proxy is designed to cache datasets on filesystems near the execution resources.  The caching proxy requires installation of software and the running of services on the cluster.  Unprivileged campus users will be unable to run or install these services.



% transfer protocols
% HTTP

% Caching in the 
I define local caching as saving the input files on the local machine and making them available to local jobs.  Local caching is different from site caching, which is done in the OSG by Squid caches.  I define site caching as when data files are stored and available to jobs from a closer source than the original.  In most cases on the OSG, the site cache is a node inside the cluster that has both low latency and high bandwidth connections to all of the execution hosts.

%BitTorrent
We use distributed transfer to mean transfers that are not from a single source.  In my case, I will be using BitTorrent \cite{cohen2008BitTorrent}, in which a client may download parts of files from multiple sources.  Additionally, the client may make available to other clients parts of the files that have already been downloaded.

BitTorrent is a transfer protocol that is designed for peer-to-peer transfers of data over a network.  It is optimized to share large datasets between peers. The authors of \cite{wei2005scheduling} and \cite{wei2007towards} discuss scheduling tasks efficiently in peer-to-peer grids and desktop grids.  Their discussion does not take into account the network bottlenecks that are prevalent in campus cluster computing.  

In \cite{briquet2007scheduling}, the authors use scheduling, caching, and BitTorrent in order to optimize the response time for a set of tasks on a peer-to-peer environment.  They build the BitTorrent and caching mechanisms into the middleware which is installed and constantly running on all of the peers.  They do not consider the scenario of opportunistic and limited access to resources.  Their cluster size is statically set, and therefore may not see the variances that users of campus clusters may see.



\section{Implementation}

The HTCondor CacheD is a daemon that runs on both the execution host and the submitter.  For my purposes, a cache is defined as an immutable set of files that has metadata associated with it.  The metadata can include a cache expiration time, as well as ownership and acceptable transfer methods.

The CacheD is designed to interface with other daemons as well as the user.  The CacheD communicates only over sockets (whether file sockets or network sockets, it doesn't matter). The communication protocol is over ClassAds \cite{raman1998matchmaking}, a key value data structure.  

\subsection{Interface and API}
To simplify communication with the CacheD, clients may use an API which will communicate with the CacheD using ClassAds.  Further, Python bindings to the CacheD C++ API were written in order to make communicating with the daemon even easier.  Figure \ref{fig:createcacheexamplecode} shows the example code necessary to create a cache.  As you can see, the code is very simple.

\begin{figure}[ht!]

\begin{lstlisting}[language=Python]
#!/bin/env python

import htcondor
import glob
import time
import sys

cached = htcondor.Cached()
cacheName = sys.argv[1]

try:
	cached.createCacheDir(cacheName, int(time.time())+1000)
except RuntimeError:
	print "Create cache failed"
	sys.exit(1)

input_glob = glob.glob(sys.argv[2])
print input_glob
try:
	cached.uploadFiles(cacheName, input_glob)
except:
	print "Upload files Fail"
	sys.exit(1)
\end{lstlisting}
\caption{Example Code to Create a Cache}
\label{fig:createcacheexamplecode}
\end{figure}

Creating a simple API to interact with the CacheD improves the user experience of using the CacheD.  The CacheD also uses this same API in order to communicate with other CacheDs.  For example, the call to replicate a cache is shown in Figure \ref{fig:requestlocalcall}.  All CacheD's and client use this same function.

\begin{figure}[h!]
\begin{lstlisting}[language=C,backgroundcolor=\color{grey},frame=none]
int requestLocalCache(const std::string &cached_server, const std::string &cached_name, compat_classad::ClassAd& response, CondorError& err)
\end{lstlisting}
\caption{Function to Request a Replica of the Cache}
\label{fig:requestlocalcall}
\end{figure}

The function \texttt{requestLocalCache} asks the CacheD you are communicating with to replicate the cache to itself.  It may respond with several options:

\begin{description}
	\item[\texttt{OK}:] The cache will be replicated.  The current status of the replication is in the \texttt{CacheState} attribute of the returned ClassAd \texttt{Reponse}.
	\item[\texttt{WAIT}:] If the CacheD has not decided whether or not to accept the cache, it can send a \texttt{WAIT} signal to the requester to ask again after some delay.
	\item[\texttt{REJECTED}:] If the CacheD has decided, through the policy language described in Chapter \ref{chapter:campusstoragepolicylanguage}, to not replicate the cache, the CacheD will respond with \texttt{REJECTED}
\end{description}

A CacheD may call this function in order to replicate a cache it stores to other CacheD's.  A transfer plugin may call this function in order to replicate the cache locally before downloading the cache.

Users may also use this API in order to modify the replication policies of the job, to extend the lease time of a cache, or to download the cache.  All interactions with the CacheD are available through the API.


\begin{figure}[ht]
\centering
\includegraphics[width=\textwidth]{images/DaemonLayout.pdf}
\caption{Daemon Locations}
\label{fig:daemonlayout}
\end{figure}


The CacheD follows the HTCondor design paradigm of a system of independent agents cooperating.  Each CacheD makes decisions independently of each other.  Coordination is done by CacheDs communicating and negotiating with each other.  Figure \ref{fig:daemonlayout} shows the location of daemons both on the submission host and the worker nodes.  The CacheD on the user's submit machine acts as the cache originator, discussed below.  The CacheDs on the worker nodes download the cache when requested.

Each caching daemon registers with the HTCondor Collector.  The collector serves as a catalog of available cache daemons that can be used for replication.

% Talk about the transfer plugin
In addition to the CacheD, a transfer plugin is used to perform the cache transfers in the job's sandbox.  The plugin uses an API to communicate with the local CacheD to request local replication requests to the local host.  After the cache is transferred locally, the plugin then downloads the cache to the job's working directory.

Expiration time is is used for simple cache eviction.  A user creates a cache with a specific expiration time.  After a cache has expired, a caching server may delete it to free space for other caches.  The expiration may be requested to be extended by the user.

\label{sec:cachedtransfermethods}
The CacheD supports multiple different forms of transferring data.  Using HTCondor's file transfer plugin interface, it can support pluggable file transfers.  For this paper, I will only use the BitTorrent and Direct transfer methods.  The BitTorrent method uses the libtorrent library to manage BitTorrent transfers and torrent creation.  The Direct method uses an encrypted and authenticated stream to transfer data from the source to the client.

An important concept of the caching framework is a cache originator.  The original daemon that the user uploaded their input files to is the cache originator.  The cache originator is in charge of distributing replication requests to potential nodes, as well as providing the cached files when requested.

The caching daemons interact with each other during replication requests.  A cache originator sends replication requests to remote caching daemons that match the replication policy that is set by the user.  The remote caching daemon then confirms that the cache data can be hosted on the server.  The remote cache then initiates a file transfer in order to transfer the cached data from the origin to the remote CacheD.

The receiving CacheD can deny a replication request for many reasons, including:
\begin{itemize}
\item The resource does not have the space to accommodate the cache.
\item The resource may not have the necessary bandwidth available in order to transfer the cache files.
\item The resource does not expect to be able to run the user's jobs and has determined that the cached files will not be used.
\end{itemize}

The ability of the receiving CacheD to deny a replication request follows HTCondor's independent agent model.

The policy expression language is modeled after the matchmaking language in the HTCondor system \cite{raman1998matchmaking}.  The caching daemon is matching the cache contents to a set of resources; therefore, it is natural to use HTCondor's same matchmaking language that is used to match jobs to resources.  Once a resource is determined to match the cache's policy expression, the caching daemon will contact the resource's caching daemon in order to initiate a cache replication.  The caching daemon on the remote resource is an independent agent that has the ability to deny a caching replication even after matchmaking is successful.  A full discussion of the policy language, as well as possible configurations, is discussed in Chapter \ref{chapter:campusstoragepolicylanguage}.

Libtorrent is built into the CacheD to provide native BitTorrent functionality.  The CacheD is capable of creating torrents from sets of files in a cache, as well as downloading cache files using the BitTorrent protocol.  Since this is a distributed set of caches, I will not use a static torrent tracker.  Rather, I will use a Distributed Hash Table \cite{dinger2009decentralized} and local peer discovery \cite{legout2007clustering} features of the BitTorrent protocol.  This ensures that there are no single points of failure.


% Do the command line usage

\subsection{Creation and Uploading Caches}
The user begins using the caching system by uploading a cache to their own CacheD, which then becomes the cache originator.  This is very similar to a user submitting a job to their own HTCondor SchedD.  Using the cache's metadata, the CacheD decides whether to accept or reject the cache.  If the CacheD accepts the cache, it stores the metadata into resilient storage.  The user then proceeds to upload the cache files to the CacheD.

The CacheD stores the cache files into its own storage area.  Once uploaded, the CacheD takes action to prepare the cache to be downloaded by clients.  This includes creating a BitTorrent torrent file for the cached files.  

Numerous protections are used in order to ensure proper usage of the CacheD.  The upload size is enforced to the size advertised in the metadata.  The client cannot upload more data to the CacheD than was originally agreed upon during cache creation.  Further, the ownership of the cache is stored in the metadata, and is acquired by authenticating with the client upon cache creation.  Only the owner may upload and download files from the cache directly.

A client may mark a cache as only allowing certain replication methods.  This can be useful if a user wishes to keep data private. BitTorrent doesn't offer the authorization framework to ensure privacy of caches. Users may mark the cache as only allowing Direct replications, which are encrypted and authenticated.

\subsection{Downloading Caches}
When a job starts, the CacheD begins to download the cache file using a file transfer plugin.  The cache is identified by a unique string that includes the cache's name and the cache's originator host.  The flow of replication requests is illustrated in Figure \ref{fig:replicationflow}.  The replication requests originate from the file transfer plugin, which sends the replication request to the node local CacheD.  The node local CacheD then sends the replication to its parent or the origin cache.  The propagation of replication requests are modeled after well-known caching mechanisms such as DNS.

\begin{figure}[h!t]
\centering
\includegraphics[width=0.65\textwidth]{images/CacheDownloadFlow.pdf}
\caption{Flow of Replication Requests}
\label{fig:replicationflow}
\end{figure}

Figure \ref{fig:replicationflow}'s flow can be shown in the following steps:

\begin{enumerate}
\item The plugin contacts the node local CacheD daemon on the worker node.  It requests that the cache is replicated locally in order to perform a local transfer.
\item The node local CacheD responds to the file transfer plugin with a ``wait'' signal.  The file transfer plugin polls the node local CacheD periodically to check on the replication request.
\item The local CacheD daemon propagates the cache replication request to its parent, if it exists.  If the CacheD does not have a parent it contacts the cache originator in order to initiate a cache replication.
\item If the cache is detected to be transferable with BitTorrent, the download begins immediately after receiving the cache's metadata from the parent or origin.
\item Once the cache is replicated locally, the plugin downloads the files from the local CacheD.
\end{enumerate}

All communication between CacheDs are authenticated using the regular HTCondor methods.  ClassAds are used for communication between the CacheDs so that the protocol can be expanded if needed. 

Each download is negotiated for the appropriate transfer method between the source, client, and the cache.  Each entity has its own preferences on the method of transfer.   Further discussion of this negotiation is discussed in Chapter \ref{chapter:campusstoragepolicylanguage}.

By default, the CacheD is capable of two transfer methods between CacheDs: the BitTorrent and the Direct transfer methods.  

If the transfer plugin successfully authenticates with a local CacheD, transfer methods are negotiated.  If supported, another transfer method is possible: the symbolic link (symlink) method.  The symlink method is preferred to directly downloading the cache for two reasons:
\begin{enumerate}
	\item Downloading the cache will create yet another copy of it, filling disk space on the local node.
	\item A symlink can create a nearly instantaneous transfer of the data from the cache directory to the execution directory.
\end{enumerate}

A symlink does not actually copy the data.  Instead, it creates a pointer to the data which is in another directory.  This symlink creates the possibility that the cache may be altered by the job, but this issue is largely ignored for now.  BitTorrent will not allow a modified cache to be replicated; therefore, there no chance that the altered cache will propagate to other nodes in the system.  BitTorrent provides this guarentee by checksumming all files in the cache before and after transfers.  If the checksum does not match, the file is re-downloaded.

The symlink transfer method allows near instant transfer of the cache from the CacheD to the file transfer plugin.  A symlink is created by the CacheD in the job's working directory pointing to the cache directory.  This symlink method eliminates transferring the cache to each job.

\begin{figure}[ht]
\centering
\includegraphics[width=\textwidth]{images/ReplicationBottleneck.pdf}
\caption{Cache Replication Showing Bottleneck}
\label{fig:cachebottleneck}
\end{figure}

In Figure \ref{fig:cachebottleneck}, you can see a traditional configuration of a cluster.  The configuration shows that there is a Network Address Translation bottleneck or a network bottleneck between the submit machine and the execution nodes.  The bottleneck limits the bandwidth between the submit machine and the execution nodes.



% Discuss the possiblity of jobs modifying the cache?

\subsection{Reporting of Replicas}

Once the cache has replicated to a CacheD, it will periodically report the replica to the origin CacheD.  The replica locations are stored in a double hash table keyed by the cache name, then the hostname.  Finally, the value is the time of the last report.  The updates are transferred in with ClassAds from the replica CacheDs.

The design of the data structure is to make lookups of cache locations very fast.  In Chapter \ref{chapter:campusstoragepolicylanguage}, I discuss possible uses of this data structure when determining where the replica should be placed.  



\subsection{Parenting of CacheDs} \label{sec:cachedparenting}
During testing of the CacheD, it was apparent that BitTorrent increases the IO queue on the host server significantly, degrading the IO performance for all jobs on the server.  This increased IO queue leads to competition between BitTorrent-enabled CacheDs on the same host.  In order to address the increased IO queue, each CacheD will designate a single daemon on the host that downloads the files through BitTorrent.  All other CacheDs will then download the cache from the parent using Direct file transfer mechanisms.  

A CacheD will discover a node local parent by querying the HTCondor Collector that holds a catalog of all CacheDs known to the system.  Figure \ref{fig:daemonlayout} illistrates the layout of the HTCondor Collector in respect to the worker nodes.  If it discovers a CacheD on the same node as itself, the CacheD started first will be chosen.  By choosing a parent that is older, it improves the chance that the parent CacheD may already have the cache downloaded.  If multiple CacheD's have the same starting time, they are alphabetically ordered by their unique name, and the first CacheD alphabetically is chosen as parent.

Parenting can also be used to create a hierarchy of caching.  This is especially useful for direct transfer mechanisms.  Figure \ref{fig:cacheparenting} shows an example of caching parenting on different clusters.

\begin{figure}[ht]
\centering
\includegraphics[width=0.9\textwidth]{images/CacheDParenting.pdf}
\caption{Illustration of Cache Parenting}
\label{fig:cacheparenting}
\end{figure}

A CacheD's parent is discovered through the configuration.  When a CacheD starts, it checks the configuration for a special attribute, the \texttt{CACHED\_PARENT}.  It then attempts to connect to the parent to verify that the parent is functional.  If the parent is found to be functional and responding to queries, the child then forwards all requests it receives to the parent, just as the node local CacheD will forward all requests to the origin in Figure \ref{fig:replicationflow}.


\section{Results For Campus Cluster}

\subsection{Experimental Design}
To evaluate the solution, I will run a BLAST benchmark from UC Davis \cite{blastbenchmark}.  I chose a BLAST benchmark due to many factors.  BLAST is used frequently on campuses, but used infrequently on clusters due to the size of the database. BLAST has very large databases that are required by each job.  This makes it difficult to use on distributed resources since each job requires significant data.
BLAST databases are frequently updated, making them poor candidates for static caching, but good candidates for short-term caching, for which the CacheD specializes.

The BLAST database distributed with the benchmark is a subset of the Nucleotide NR database.  In the tests, I will use a larger subset of the NR database in order to demonstrate the efficiency of the solution.

For researchers, the time to results is likely the most important metric.  The stage-in time of data can be a large component of the entire workflow time.  I will measure the time for stage-ins as well as the average stage-in time.

I designed two experiments that represent my experience on campus infrastructure.  In the first experiment, I will allow 100 simultaneous jobs to start at the same time and measure the average download time versus the number of distinct nodes.  This experiment also includes the download time for child caches.  I chose 100 jobs somewhat arbitrarily in order to completely fill all of the nodes I was allocated on the cluster.  

In the second experiment, I compared the total stage-in time for a variable number of jobs while number of distinct nodes remains constant at 50.  This will show that the cache is working to eliminate transfer times when the files are already on the node.  Further, it will compare HTCondor's File Transfer method versus the CacheD's two transfer methods: BitTorrent and Direct.

When the number of jobs is fewer than 50, each job must download the cache since there are 50 nodes available for execution.  When the number of jobs is more than 50, all jobs that run after the initial download use a cached version of the data.

In the experiments, each job will use the CacheD to stage-in data to the worker nodes.  The jobs will be submitted with glideins created by Bosco \cite{weitzel2014accessing}  and the Campus Factory \cite{weitzel2011campus}.  Bosco allows for remote submission to campus resources while the Campus Factory allows for on-demand glidein overlay of remote resources.  The Campus Factory is used in order to create and run glideins which, in turn, run the CacheD daemon.  Bosco was used in order to submit to multiple campus resources simultaneously.

These two experiments were conducted on a production cluster at the Holland Computing Center at the University of Nebraska--Lincoln (UNL).

\subsection{Results}

I completed 41 runs of the BitTorrent versus Direct transfer  experiments on the UNL production cluster.  I first confirmed my suspicion that the Direct transfer method would result in a linear increase in the average stage-in time to transfer the cache as I increased the number of distinct nodes.  Conversely, I found that the BitTorrent transfer method did not significantly increase the average stage-in time as I increased the number of distinct nodes.  The BitTorrent transfer method was faster than the Direct in all experiments.


\begin{figure*}[h!t]
\centering
\includegraphics[width=\textwidth]{images/CombinedPlot.pdf}
\caption{Comparison of Direct and BitTorrent Transfer Methods with Increasing Distinct Node Counts}
\label{fig:combinedgraph}
\end{figure*}

Figure \ref{fig:combinedgraph} shows that the BitTorrent transfer method is superior to Direct for all experiments that were run.  Since multiple CacheDs on the same node will parent to a single CacheD, the number of distinct nodes is the dependent variable.  After the parent cache downloads the cache for the node, then each child cache will download from the parent using the Direct transfer method.

The Direct method of transfer follows a linearly increasing time to download the cache files.  This can be explained by bottlenecks of the transfers between the host machine and the execution nodes.  The increase in number of distinct nodes increases the stage-in time for any individual node.

The average download times for BitTorrent stage-ins are also shown in Figure \ref{fig:combinedgraph}.  The stage-in time does not significantly increase as the distinct nodes increases.  This met my expectations.  I expect this trend to continue as the number of distinct nodes increases since BitTorrent can use peers to speed up download time.

\begin{figure}[h!t]
\centering
\includegraphics[width=0.8\textwidth]{images/modes_vs_downloadtimes-grayscale.png}
\caption{Historgram of Transfers Mode vs. Download Times}
\label{fig:histmethod}
% probe-output.03.13.2015.2
\end{figure}

To better illustrate how parenting affects the download time of a cache, I show a histogram of the different modes in Figure \ref{fig:histmethod}.  The figure shows that while the parents download first, and nearly at all the same time, the children take a variable amount of time to download.  This variability can be attributed to the number of children on a node.  The more children downloading the cache at the same time, the slower each download will take. 

 

\begin{figure}[h!t]
\centering
\includegraphics[width=\textwidth]{images/BlastRunOverview.pdf}
\caption{Timeline of Blast Runs}
\label{fig:timelineblastruns}
\end{figure}

The observed behavior of the CacheD timeline is shown in Figure \ref{fig:timelineblastruns}.  If there are several children on a node, then the CacheD will wait for the parent to download the cache, then each child will download from the parent.  The parent will begin the BLAST job immediately after downloading the cache.  

Disk contention was observed on the nodes while the children were downloading the cache and the parent was running BLAST.  This disk contention warrants further investigation.  Multiple copies of the cache will reside on the same node, but this is necessary since the CacheD is running on opportunistic resources.  At any time, a parent or child may be preempted, and their copy of the cache will be removed.  An independent copy of a cache for each CacheD will guarantee that the cache will survive as long as the CacheD, and the cache will be available for subsequent executions.

For the second experiment, I calculated the total stage-in time for a variable number of jobs.

\begin{figure}[h!t]
\centering
\includegraphics[width=\textwidth]{images/StageinPlot.pdf}
\caption{Transfer Method vs Number of Jobs}
\label{fig:methodvsnumjobs}
\end{figure}

When I limit the number of nodes to 50, I can clearly see the effect of the caching by varying the number of jobs.  In Figure \ref{fig:methodvsnumjobs}, both the Direct and BitTorrent transfer methods have a natural bend at about 50 jobs.  This correlates to when the CacheD has on-disk caches of the datasets, and the transfer to the job's sandbox is nearly instantaneous.  

The HTCondor file transfer method has a shorter stage-in time for low numbers of distinct nodes than the Direct method.  This can be explained by the increased overhead that the CacheD introduces when transferring datasets.  After all 50 nodes have the dataset cached locally, the Direct transfer method becomes more efficient than the HTCondor file transfers.

\subsubsection{BitTorrent Behavior}

To verify that BitTorrent was working as expected on the campus, I captured each block (the smallest unit of transfer in the BitTorrent protocol) from the source to the destination.  I then graphed the resulting transfer links.  For this experiment, I only submitted five jobs in order to limit the size of the graph, and to improve readability.  I used the same 15 GB NR database as before.

\begin{figure}[h!t]
	
\centering
\includegraphics[width=\textwidth]{images/verbose_group.png}
\caption{Graph of Transfer Nodes in the BitTorrent Transfer}
\label{fig:bittorrenttransfernodes}
	
\end{figure}

Figure \ref{fig:bittorrenttransfernodes} shows the transfers between the nodes.  The libtorrent library that is used by the CacheD to implement the BitTorrent protocol allows for alerts to be propagated each time a block is transfered.  The CacheD periodically polls the library for alerts and prints any block movement activity.  For this graph, I was moving the 15 GB subset of the NR database mentioned in the experimental design.  It is expected that I am not able to capture every block transferred between the nodes, since the alert buffer may overflow and further alerts will be lost until the alert buffer is cleared by the periodic check.

Scanning software was written to scan the debug output from the CacheD for the block movement data.  It was then transformed into a DOT file that could be transformed with the GraphViz \cite{ellson2002graphviz} application.  The nodes are identified by their unique BitTorrent assigned identifiers, except the origin, which is signified by \textit{Cache Origin}.  The \textit{Cache Origin} is the original server with the cache before the BitTorrent transfers replicate it to other nodes.

From Figure \ref{fig:bittorrenttransfernodes}, you can notice several interesting points.  First, even though the cache must be transferred to all five nodes of the system, the \textit{Cache Origin} is shown as only transferring the cache approximately one time, and nearly equally to two different nodes.  Once the cache is in the system, the \textit{Cache Origin} does not transfer the data to any of the other nodes.  

Once the cache is fully inside the system, the CacheDs transfer data only between each other inside the cluster.  The libtorrent library detects ``fast'' nodes and preferentially transfers with them.  Since nodes inside the cluster are near each other on the network and are on an HPC system with a high performance network interconnect, the library highly prefers transferring from only the cluster nodes.  You can see that the two original nodes transfer the cache to the other nodes in the system, including each other.

%it begins with less overhead than the direct or the BitTorrent methods at low job counts.  But as the number of jobs increase, so to does the total stage-in time.  Since the condor file transfer method does not cache the data, it continues to linearly increase in stage-in time as the number of jobs increase.

\section{Results for Open Science Grid}

\subsection{Experimental Design}
For the OSG testing, I again used Bosco.  Bosco was connected to the Holland Computing Center's (HCC) GlideinWMS frontend which will run the Bosco glideins on worker nodes on the OSG.  The GlideinWMS system for HCC submits to about 20 sites on the OSG.  Not all sites were available to run jobs during the experiments.

\begin{figure}[ht]
	\centering
	\includegraphics[width=\textwidth]{images/OSGDaemonLayout.pdf}
	\caption{OSG Daemon Locations, Described Below}
	\label{fig:osgdaemonlayoutcached}
\end{figure}

Figure \ref{fig:osgdaemonlayoutcached} shows the daemon layout when running on the Open Science Grid.  The flow of the experimental BLAST jobs are:

\begin{enumerate}
	\item The user submits jobs to Bosco using standard HTCondor commands.
	
	\item Bosco detects idle jobs on the Bosco system, and deploys Bosco glideins to the connected HCC GlideinWMS Frontend in order to service the idle jobs.  Bosco transfers the worker node binaries to the HCC GlideinWMS Frontend to be run as the jobs.
	
	\item The GlideinWMS glidein starts on the remote OSG cluster.  It communicates back with the GlideinWMS Frontend in order to retrieve the job, which is a Bosco glidein.
	
	\item The Bosco glidein starts and reports back to the Bosco system on the user's submit machine.  The submit machine then will send the BLAST job to the remote Bosco glidein running on the OSG cluster.
	
	\item At the same time that the Bosco glidein reports back to the Bosco system, the CacheD is reporting to the HTCondor Collector on the user's submit machine.  This will add the CacheD to the list of known CacheDs in the system.
\end{enumerate}

For each experiment, I submitted 100 BLAST jobs at a time.  I compared the BitTorrent transfer method to that of the leading transfer method on the OSG, HTTP Caching. 

The HTTP caching jobs pulled the BLAST database through HTTP from the job submission server.  Figure \ref{fig:httpcachearchitecture} shows the architecture of HTTP caching on the OSG.  The HTTP request is cached through a Squid server near the execution host.

\begin{figure}[ht]
	\centering
	\includegraphics[width=\textwidth]{images/HTTPCache.pdf}
	\caption{HTTP Cache Architecture}
	\label{fig:httpcachearchitecture}
\end{figure}

The HTTP requests have no method to parent to one another; therefore, I will not compare the number of parents for HTTP compared to BitTorrent as I did for the campus experimental runs.  Instead, I will compare the average time to complete the transfer of the input data files.

I again used the same data source node as in the campus experiments.  It has a 1 Gbps connection to the internet.

Since the CacheD is able to cache the input data on the node, and I have shown in the campus experiments that the CacheD is nearly instantaneous in transferring the cached data to the job directory, I will not compare subsequent total stagein times as I did in the campus experiments.

\subsection{Results}

While processing the results from the experimental runs, I noticed that the vast majority of CacheDs where running in parent mode.  In the results, I saw between 71 and 96 parents out of 100 jobs.

Indeed, I experienced so many parents, that I was unable to complete any Direct transfer experiments.  The transfers took too long to complete and the jobs where evicted from the remote OSG clusters before the could complete the transfer of the BLAST databases.  Therefore, I will not test the Direct transfer method on the OSG.

\begin{table}[h!t]
	\centering
	\bgroup
	\def\arraystretch{1.5}
\begin{tabular}{|l|r|r|}
\hline
\textbf{Transfer Type} & \textbf{Average Time (minutes)} & \textbf{Average MB/s} \\ \hline
BitTorrent & 17.77 & 13.66 \\ \hline
HTTP & 52.17 & 4.65 \\ \hline
\end{tabular}
\egroup
\caption{Transfer Times for Different Transfer Type}
\label{tbl:osgtransferstats}
\end{table}

As you can see in Table \ref{tbl:osgtransferstats}, the Bittorrent method is significantly faster than the HTTP method.  

I can also compare all of the transfer methods across both the campus and OSG experiments.  I show this comparison in a Violin Plot \cite{hintze1998violin} in Figure \ref{fig:violinplots}.  The violin plot shows a probability distribution of transfer speeds for the different transfer methods.  I ran each experiment at least 25 times, but as much as 43 times in the case of OSG BitTorrent runs.

\begin{figure}[h!t]
\includegraphics[width=\textwidth]{images/ViolinPlot.pdf}
\caption{Violin Plot of the Transfer Speed Comparing Transfer Speeds}
\label{fig:violinplots}
\end{figure}

Figure \ref{fig:violinplots} clearly shows that the BitTorrent methods are faster at transferring the experimental data than the Direct or HTTP methods.  Even though the HTTP and Direct methods look similar in speed, they are comparing different aspects of the transfers.  Direct method has sometimes as much as half as many downloaders (parents) as the HTTP method, in which all of the downloaders are parents.  Therefore, given the same number of downloaders, HTTP should be faster than direct.

The BitTorrent method for the campus has a much wider variance of transfer speeds than any other transfer method.  This can be explained by the large variance in the number of parents available to download, compared to the other transfer methods.  The BitTorrent method for the OSG is almost exclusively faster than the HTTP method on the OSG.

\subsection{Aggregate Bandwidth}

While running the CacheD experiments, I discovered that the average bandwidth for transfers was much higher that of the source node.  Therefore, I set out to compare the aggregate bandwidth of different transfer methods.

\begin{figure*}[h!t]
\centering
\subfloat[Campus Direct\label{fig:aggregatecampusdirect}]{\includegraphics[width=0.5\textwidth]{images/campus-aggregatedirect.png}}
\subfloat[Campus BitTorrent\label{fig:aggregatecampusbittorrent}]{\includegraphics[width=0.5\textwidth]{images/crane-agreggatebittorrent.png}} \\
\subfloat[OSG HTTP\label{fig:aggregateosghttp}]{\includegraphics[width=0.5\textwidth]{images/osg-aggregatehttp.png}}
\subfloat[OSG BitTorrent\label{fig:aggregateosgbittorrent}]{\includegraphics[width=0.5\textwidth]{images/osg-aggregatebittorrent2.png}}

\caption{Comparison of Aggregate Bandwidth During Transfers}
\label{fig:aggregatebandwidthcached}
	
\end{figure*}

The aggregate bandwidth shown in Figure \ref{fig:aggregatebandwidthcached} is calculated from the historical logs of the experimental runs.  For each transfer, the average bandwidth is assumed to start at the transfer start time and end at the transfer end time.  Then, each of these transfers is overlay on top of each other.  Each of the graphs in the Figure are a single experimental run, but are typical for the transfers.  The origin server's available bandwidth is also shown on the graphs as a horizontal line at 1 Gbps.

From Figure \ref{fig:aggregatebandwidthcached}, you can see the difference between the different transfer methods.  The Campus Direct method in Figure \ref{fig:aggregatecampusdirect} has the lowest aggregate bandwidth of all of the transfer methods, while BitTorrent on the campus in \ref{fig:aggregatecampusbittorrent} has the highest.  The campus BitTorrent method multiplies the available bandwidth of the origin server by over 12 times.  This factor of 12 can be attributed to servers in the cluster distributing data between with the BitTorrent protocol without transferring it from the origin server.  Using the Direct method, the CacheD must transfer the data from either the origin or a parent on the local node.

For the two OSG transfer methods \ref{fig:aggregateosghttp} and \ref{fig:aggregateosgbittorrent}, you can see that the HTTP method almost reaches the same aggregate bandwidth as the BitTorrent method, nearly 10 Gbps.  Both transfer methods have long tails after a large peak in transfer speed.  But the BitTorrent method maintains the peak of transfer speed until roughly 1200 seconds into the transfer.  The HTTP method's peak drops significantly before 800 seconds into the transfer.  The HTTP also has significant transfers yet to be completed, hence the not only long tail, but high transfer speed of the tail.  Upon further investigation, I found that the HTTP performance differed significantly between sites.  Table \ref{tbl:transferspeedsites} shows the transfer speeds of sites in the same run.

\begin{table}[h!t]
	\centering
	\bgroup
	\def\arraystretch{1.5}
	\begin{tabular}{l|r}
\textbf{OSG Site} & \textbf{Average Transfer Speed} \\ \hline
UCSD & 236 Mbps \\ \hline
Nebraska & 116.8 Mbps \\ \hline
AGLT2 & 104 Mbps \\ \hline
Wisconsin & 91.2 Mbps \\ \hline
UChicago & 90.4 Mbps \\ \hline
SPRACE & 80.8 Mbps \\ \hline
MIT & 26.4 Mbps \\ \hline
	\end{tabular}
	\egroup
	\caption{HTTP Transfer Speeds by Site}
	\label{tbl:transferspeedsites}
\end{table}

In Table \ref{tbl:transferspeedsites}, you can see the disparity in transfer speeds between sites.  University of California, San Diego (UCSD) is twice as fast downloading input HTTP data than the next highest, Nebraska.  But MIT was 10 times as slow as UCSD.  This can be attributed to the speed and configuration of the HTTP caching servers at the different sites.  UCSD has a HTTP caching server with a 10 Gigabit connection and a solid state disk.  This enables very fast transfers from the caching server to the worker nodes.


\begin{table}[h!t]
\centering
\bgroup
\def\arraystretch{1.5}
\begin{tabular}{l|r|r}
	\textbf{OSG Site} & \textbf{Average Transfer Speed} & \textbf{Percent Change} \\ \hline
	UCSD & 169.6 Mbps & -28.14 \% \\ \hline
	NU Crane & 134.4 Mbps & NA \\ \hline
	Northwestern & 133.6 Mbps & NA \\ \hline
	Nebraska & 129.6 Mbps & +10.96 \% \\ \hline
	Purdue & 124 Mbps & NA \\ \hline
	MIT & 113.6 Mbps & +330.3 \% \\ \hline
	Michigan & 104.8 Mbps & +0.77 \% \\ \hline
	Wisconsin & 94.4 Mbps & +3.5 \% \\ \hline
	UChicago & 88 Mbps & -2.69 \% \\ \hline
	Brookhaven & 67.2 Mbps & NA \\ \hline
	Connecticut & 54.4 Mbps & NA \\ \hline
	
\end{tabular}
\egroup
\caption{BitTorrent Transfer Speeds by Site}
\label{tbl:bittorrenttransferspeedsites}
\end{table}

Table \ref{tbl:bittorrenttransferspeedsites} shows the BitTorrent transfer speeds from the same experiment shown in \ref{fig:aggregateosgbittorrent}.  As you can see, there are more sites involved downloading the cache than the HTTP transfer method.  But you can see that the transfers are usually faster than the HTTP method.  In addition, there are more sites that are faster than the  HTTP method.

It is not unusual to run on very different sites in separate experiments on the OSG.  The BitTorrent and the HTTP experiments were run around 20 days apart.

When comparing the campus and OSG transfer, you can notice long tails as the transfer speeds slowly decline.  This can happen for many reasons.  For example, for the OSG HTTP method, it occured because a single cluster had a very slow transfer speed, which slowed the download for all jobs running at that cluster.

The long tail of the OSG BitTorrent method are very slow transfers.  After the initial very fast BitTorrent transfers,  then children CacheDs begin their download of the cache.  These downloads of the cache from the parent are slower than the BitTorrent downloads because they are coming from a single source, and multiple children may be downloading at the same time.  To further look at the OSG BitTorrent method, I graphed the download time by the mode (either Cached, Parent, or Child) in Figure \ref{fig:dowloadmodebittorrent}.

\begin{figure}[h!t]
\centering
\includegraphics[width=0.8\textwidth]{images/osg-aggregatebittorrentmodes-grayscale.png}
\caption{Download Time by Mode for a Single OSG BitTorrent}
\label{fig:dowloadmodebittorrent}
% probe-output.05.05.2015.3
\end{figure}

Figure \ref{fig:dowloadmodebittorrent} shows the download method time by mode.  As you can see in the graph, the vast majority of downloads are from parents.  But some children are also interleaved with the parent downloads.  The final download is a singular child download.  This child download may have gone slow because of inadequate disk bandwidth on the local node.  This inadequate disk bandwidth could be exacerbated by the parent CacheD running BLAST while the child is still downloading the cache.



\section{Conclusions}
I have presented the HTCondor CacheD, a technique to decrease the stage-in time for large shared input datasets.  The experiments proved that the CacheD decreases stage-in time for these datasets.  Additionally, the transfer method that the CacheD used can significantly affect the stage-in time of the jobs.

The BitTorrent transfer method proved to be a efficient method to transfer caches from the originator to the execution hosts.  In fact, the transfer time for jobs did not increase as the number of distinct nodes requesting the data increased.  Any bottlenecks that surround the cluster are therefore irrelevant using the BitTorrent transfer method.

I investigated OSG transfers for both HTTP and BitTorrent.  I found that not all sites have equivalent HTTP caching setups.  For example, one site was three times faster than the second most 

In addition, I found that the CacheD using the BitTorrent transfer method out-performed the popular HTTP transfer method on the OSG.  Further investigation of slow transfers must be completed in order to further optimize the BitTorrent transfers on the OSG.  A possible solution could be to give up on the transfer after some timeout or if the transfer speed is too slow.  Although this timeout and transfer speed thresholds would be difficult to set accurately.

%For caching methods that attempt to optimize per cluster access, such as HTTP proxy methods, the results would like be very similar to those shown above.  Per cluster caching still bottlenecks the transfers to a single or set of nodes near the cluster.  They are better for optimizing latency of small accesses rather than aggregate bandwidth, which is required for large input datasets.

In the future, I plan to investigate incorporating job matchmaking with cache placement.  The HTCondor Negotiator could attempt to match jobs first against resources that have the input files before matching against any available computing resources.






\chapter{Campus Storage Policy Language}
\label{chapter:campusstoragepolicylanguage}

\section{Introduction}

In the previous chapter, the HTCondor CacheD was introduced and benchmarked.  In this section, we will discuss the policy framework that allows the CacheD to represent heterogeneous resources on campus or cyberinfrastructure resources.  The CacheD's policy framework is used whenever it interacts with another CacheD.  This policy framework allows the CacheD to act as an independent agent within the distributed system.

In a distributed computing system, independent agents are designed to act on behalf of entities such as users, hosts, or entire clusters.  The agents attempt to fulfill the goals of the entities that they represent, even in the chaotic environment characteristic of a distributed computing system.  In order to fulfill these goals, the agents must know and understand them.  Therefore, a policy language exists to express the goals of the entity.

The policy language must be flexible enough in order to express and follow the goals and instructions of the users.  The goals of the policy language are:

\begin{enumerate}
	\item To express attributes of the entities such as the cache, CacheDs, and the host.
	\item To write policies, taking into account the attributes of the entities.
	\item To be easy to read and write the expressions.
	\item To Allow users to define their semantics and attributes in a schema-free manner.
\end{enumerate}



We implement this policy language in the CacheD using HTCondor ClassAds \cite{raman1998matchmaking}.  ClassAds are an independent library developed by the HTCondor project and are used for communication between HTCondor components.  They provide all of the attributes described above.

\begin{enumerate}
	\item The Key-Value structure of ClassAds allow for attributes of the entities to be expressed in strings, values, lists of string, or expressions.
	\item Expressions can reference attributes in the current and the matching ClassAds.
	\item Expressions are written as expressions with semantics familiar to most programmers.
\end{enumerate}

The CacheD has a few interaction points when it must interact with other agents, such as other CacheDs or users.  Those interaction points are:
\begin{description}
	\item[Choosing a Replication Target] - A CacheD that is the origin to a cache may choose to proactively replicate to other CacheDs.  Choosing a replication target requires matching the cache's requirements with that of the target CacheD's.
	\item[Accepting Cache Replication] - A CacheD must decide if it can accept a cache when it receives a replication request.  This decision is based on its own policy, as well as attributes of the incoming cache.
	\item[Transfer Method] - The transfer method for a cache to be replicated is chosen after a cache has been accepted.  This is a prioritized list of acceptable transfer methods for the cache.

\end{description}



\section{Policy Language}
The policy language used by the CacheD is the HTCondor ClassAds \cite{raman1998matchmaking}.  ClassAds offer the flexibility to describe resources with attributes. 

% example CacheD ClassAd
\begin{figure}
\begin{lstlisting}
CachedServer = true
Machine = "red-foreman.unl.edu"
LastHeardFrom = 1433790880
UpdatesTotal = 8660
Name = "cached-22815@red-foreman.unl.edu"
CondorPlatform = "$CondorPlatform: X86_64-ScientificLinux_6.5 $"
UpdatesHistory = "0x00000000000000000000000000000000"
UpdatesLost = 0
TotalDisk = 6769920
UpdateSequenceNumber = 32307
UpdatesSequenced = 8659
MyAddress = "<129.93.239.170:11000?noUDP&sock=22815_fb39>"
AuthenticatedIdentity = "dweitzel@unl.edu"
DetectedMemory = 7807
Requirements = MY.TotalDisk > TARGET.DiskUsage
CondorVersion = "$CondorVersion: 8.3.1 Dec 22 2014 BuildID: UW_development PRE-RELEASE-UWCS $"
DetectedCpus = 2
DaemonStartTime = 1431839398
CurrentTime = time()
MyCurrentTime = 1433790880
\end{lstlisting}
\caption{CacheD ClassAd Example}
\label{lst:cachedclassad}
\end{figure}

Figure \ref{lst:cachedclassad} shows an example of a CacheD's ClassAd.  The attributes describe the CacheD daemon and the host it runs on.  For example, the \texttt{DaemonStartTime} is a representation of when the daemon started.  \texttt{TotalDisk} describes how much disk is available on the host where the CacheD is running.

% How matching works
Matching of ClassAds is done by comparing attributes between two sets of ClassAds.  The attribute \texttt{Requirements} takes a special meaning when matching two ClassAds.  The 
\texttt{Requirements} attribute is a boolean expression that is evaluated in the context of both the current ClassAd and the matching ClassAd.  In the example in Figure \ref{lst:cachedclassad}, the CacheD's ClassAd would only match another ClassAd if the expression is \texttt{MY.TotalDisk > TARGET.DiskUsage}.  This means that the CacheD will only accept caches that are smaller in size than the available disk on the host.  \texttt{MY} and \texttt{TARGET} refer to the current ClassAd and the matching ClassAd, respectively.


The \texttt{Requirements} attribute in Figure \ref{lst:cachedclassad} references other attributes in both the current ClassAd and the matching ClassAd.  Attributes can reference other attributes in order to form strings, lists, or boolean expressions.  In this example, the \texttt{Requirements} attribute references other attributes in order to create a boolean expression.

\subsection{Extending CacheD Attributes}

% Extending policy language
The ClassAd describing the CacheD can be extended by using the CacheD Cron mechanism.  The CacheD Cron executes an external program in order to collect statistics and report the results in the CacheD's ClassAd.  These statistics can then be used to better describe either the daemon or the host machine.  The CacheD Cron is configured by specifying the job's attributes in the HTCondor configuration.

\begin{figure}[h!t]
\begin{lstlisting}
CACHED_CRON_CONFIG_VAL = $(RELEASE_DIR)/bin/condor_config_val
CACHED_CRON_JOBLIST = $(STARTD_CRON_JOBLIST) test
CACHED_CRON_TEST_MODE = Periodic
CACHED_CRON_TEST_EXECUTABLE = $(RELEASE_DIR)/test.sh
CACHED_CRON_TEST_PERIOD = 15s
\end{lstlisting}
\caption{CacheD Cron Configuration}
\label{lst:cachedcronconfiguration}
\end{figure}

Figure \ref{lst:cachedcronconfiguration} shows the configuration in order for the CacheD to periodically run a program named ``test.''  The executable, \texttt{test.sh}, will run tests and output a ClassAd that will be merged into the CacheD's ClassAd.  It will be run  at periods of every 15 seconds.


\begin{figure}
\begin{lstlisting}
TestResult = 100
TestRan = TRUE
TestHost = "hostname.unl.edu"
\end{lstlisting}
\caption{Example Output from CacheD Cron: \texttt{test.sh}}
\label{lst:cachedcronoutput}
\end{figure}

The output of the test executable is ClassAds that will be injected into the daemon.  Figure \ref{lst:cachedcronoutput} shows the example output from running the test program.  In this output, it sets three attributes, an integer, a boolean value, and a string.  

A example of using the CacheD Cron is to measure the IO operations per second that a host is able to complete.  This information can be used to better match caches with machines which can run the applications.  
%The period of testing the IO capabilities of a node should be longer than the 15 seconds shown in Figure \ref{lst:cachedcronconfiguration}.


\section{Uses of the Policy Language in the CacheD}

The CacheD uses the ClassAd policy language when communicating with other daemons.  For each interaction, the CacheD must make a decision, and therefore relies on the ClassAd policies in order to decide whether to perform an action.  Each of these actions are described briefly in the introduction to this chapter.  We will now discuss the details of those interactions and choices the CacheD may make in the next few sections.


\subsection{Choosing a Replication Target}
% when it is used, and by who
Each cache has a single ``origin CacheD.''  This CacheD is the CacheD where the user initially uploaded the cache.  This origin CacheD has the option to proactively replicate the cache without it being requested by jobs.  The data transfer can occur while another job is currently being run.

% What is it matching against
The origin CacheD will periodically query the HTCondor Collector to receive a list of CacheD ClassAds.  Then, the origin will iterate through each of these ClassAds, attempting to match the cache with a CacheD.

For each ClassAd from the cache and a remote CacheD, the origin will attempt a mutual match.  Therefore, the cache must accept the CacheD, and the CacheD must accept the cache.  The \texttt{Requirements} expression is evaluated for both of the ClassAds.  The default basic \texttt{Requirements} expression is to require that the CacheD has enough disk space for the cache.

For the CacheD, the default \texttt{Requirements} are:
\begin{lstlisting}
Requirements = MY.TotalDisk > TARGET.DiskUsage
\end{lstlisting}

And for an uploaded cache, the default \texttt{Requirements} are:
\begin{lstlisting}
Requirements = MY.DiskUsage < TARGET.TotalDisk
\end{lstlisting}

\texttt{MY} refers to attributes in the current ClassAd, while \texttt{TARGET} refers to attributes in the matching ClassAd.  \texttt{TotalDisk} is the amount of disk available to a CacheD.  \texttt{DiskUsage} is the total file size of the cache.  An example value of the \texttt{TotalDisk} can be seen in the example ClassAd of a CacheD shown above in Figure \ref{lst:cachedclassad} on page \pageref{lst:cachedclassad}.

% What if it does match

If the cache and remote CacheD match, the origin CacheD will send a cache replication request to the remote CacheD.  The remote CacheD will then decide if it will accept the replication request.

% What if it does not match
If the two do not match, then the origin server will not send a replication request to the remote CacheD.

% Special attributes
Additionally, there are special attributes available during this matching.  One special attribute used during some of the experiments is the \texttt{CacheRequested} attribute.  This attribute is set to the boolean \texttt{TRUE} when the cache is requested by a job.  When the cache is requested by a origin CacheD replication request, it is set to \texttt{FALSE}.  This attribute can be used in a cache's \texttt{Requirements} expression to limit cache replication to only those nodes that have jobs that have requested the cache.  An example expression would be:
\begin{lstlisting}
Requirements = (MY.DiskUsage < TARGET.TotalDisk) && (TARGET.CacheRequested =?= true)
\end{lstlisting}

Further special attributes are planned, such as an attribute whose value is the number of replications of the cache already completed.  This can be used to limit the number of CacheDs that have the cache.

\begin{table}[h!t]
	\centering
	\bgroup
	\def\arraystretch{1.5}
\begin{tabular}{l | p{10cm}}
\textbf{Attribute} & \textbf{Use} \\ \hline
\texttt{CacheRequested} & Boolean cache property set to \texttt{TRUE} when the cache has been requested by a job. \\
\texttt{CacheReplicas} & Cache property available on the cache origin.  A numerical value representing the number of complete replicas stored by CacheDs. \\
\texttt{DiskUsage} & Size, in kilobytes, of the cache. \\
\texttt{TotalDisk} & Available disk, in kilobytes, to store caches. \\
\texttt{BandwidthUsed} & Bandwidth used on the CacheD host, in Gbps. \\
\texttt{CacheDIops} & CacheD property of current measured IOPS on the host. \\
\texttt{ActiveTransfers} & Number of active transfers on the CacheD.
\end{tabular}	
	\egroup
	\caption{Sample CacheD Attributes}
	\label{tbl:cachedattributes}
\end{table}

\subsection{Accepting Cache Replication}
% When it is used, and by who
A CacheD can receive cache replication requests from three sources:

\begin{enumerate}
	\item An origin CacheD sending out proactive replication requests.
	\item A job requesting a cache.
	\item A child CacheD (as described in section \ref{sec:cachedparenting}).
\end{enumerate}

In each of these requests, the receiving CacheD has the choice to accept the replication or deny it.  When it receives a cache replication request, it looks up the cache's ClassAd and does mutual matching with its own ClassAd.  This is similar to the mutual matching done when an origin CacheD is issuing proactive replication requests.  It is important to re-run this mutual matching in case the CacheD's state has changed.  The CacheD's state could change if it has downloaded a large cache, therefore, altering the available disk for additional caches.

If the CacheD's mutual matching with the cache's ClassAd is successful, then the CacheD will accept the cache and begin negotiating transfer methods.  If the CacheD and the cache's ClassAd do not match, then the CacheD will reject the cache.

The approval or rejection of the cache is done asynchronously from the request for replication.  Therefore, the CacheD keeps a data structure of rejected caches (accepted caches are kept in the local cache database).  When the client next asks for the replication status of the cache, the CacheD will respond with the accept or reject status.  The cached rejection request expires after 15 minutes.


\subsection{Transfer Method}

% Who uses it
Each cache has a list of acceptable transfer methods.  A user may set this list of acceptable transfer methods when uploading the cache.  This list is priority ordered, with the preferred transfer method listed first.

% List formation
\begin{figure}
\begin{lstlisting}
ReplicationMethods = "BITTORRENT, DIRECT"
\end{lstlisting}
\caption{Example Replication Method for a Cache}
\label{lst:cachetransferlist}
\end{figure}

In the example shown in Figure \ref{lst:cachetransferlist}, the cache has a preference for transferring the files over BitTorrent, but will accept the Direct transfer method if needed.  Transfer methods are described in full in Section \ref{sec:cachedtransfermethods}.  These are the only possible methods now, but may expand to other methods in the future.

% Negotiating preferences
After accepting a cache to be downloaded, the CacheD will negotiate the transfer method with the cache's ClassAd that was downloaded during the acceptance testing stage. The cache's ClassAd includes the \texttt{ReplicationMethods} attribute, which is a priority list of acceptable transfer methods.  The CacheD has its own \texttt{ReplicationMethods} that is set in its configuration.  The CacheD iterates through its own methods until it finds a matching transfer method in the cache's methods.

As an independent agent, the CacheD prefers its own transfer priority list rather than the cache's priority list.  Psuedo code for the transfer negotiation is shown in Algorithm \ref{alg:negotiatetransfer}.

\begin{algorithm}
\caption{Negotiating Transfer Method Function}
\begin{algorithmic}
\State $cacheMethods\gets cacheClassAd[ReplicationMethods]$
\State $cachedMethods\gets config(ReplicationMethods)$
\ForAll{$cachedMethod \in cachedMethods$}

\ForAll{$cacheMethod \in cacheMethods$}
\If{$cachedMethod = cacheMethod$} 
\State \Return $cachedMethod$
\EndIf


\EndFor

\EndFor
\end{algorithmic}
\label{alg:negotiatetransfer}
\end{algorithm}
	
	

%\section{Inserting Attributes into Policy}

%\subsection{Measuring Storage}
%In order to provide matchmaking for resources, the resources need to be accurately described and advertised.  This requires measuring the storage capabilities and capacity of the resources and advertising those attributes to the catalog of resources.

%The measurements must be performed on the execution targets as they are the temporary storage targets.  The execution targets will measure the storage capabilities in order to determine if the jobs can run.

%TODO: Talk about measuring storage values

\section{Example Policies}

In order to better describe how data should be moved, we must categorize the data as shared or unique, private or non-private.  This creates a 2x2 matrix of possibilities of data.  Below, we define each of these categorizations.  Each of these categories comes with its own restrictions on how the data may be moved and how it is presented to the user.

\begin{table}[h!t]
	\bgroup
	\def\arraystretch{1.5}
\begin{tabular}{l | l | l }
& \textbf{Public} & \textbf{Private} \\ 
\hline
\textbf{Shared} & Executables and Libraries & Personal Identifying Information \\ 
\hline
\textbf{Unique} & Input Parameters & BLAST Query Files \\
\hline
\end{tabular}
\egroup
\caption{Example Data Types}
\label{tbl:exampledatatypes}
\end{table}

Shared data is data that is the same for multiple jobs in a job set.   In many cases, the majority of the files in the job sandbox can be considered shared data.  Examples of shared data are job executables and libraries.  

Frequently the job executables are the same for a large number of jobs.  Since the executables are the same, contextualization of the job is done through other methods, such as arguments or parameter files.  An example application that would use the same executables and libraries are Monte Carlo \cite{binder2010monte} simulations.  In these applications, the executables are the same for every job.  Each job is given a unique identifier which is used for the starting condition for the random generation.

Experimental data could also be shared between multiple jobs in a job set.  This can include common input data such as databases.  For example, BLAST \cite{altschul1990basic} jobs require a database of sequences of proteins which are then matched with specific queries.  The database is typically the same for a large number of queries.  

We define unique data as data which is different for each job.  The unique data may be small, such as parameter files.  Or they may be large, such as sections of a database to search.  By definition, unique data is defined as data which would not benefit from shared transfer; no other job needs the same data.

For our consideration, data which is not the same for every job in a job set, but is shared between jobs in a subset of the jobs, will be considered shared data.

% private versions of shared and unique
We define private data as data which the user wants to prevent others from viewing.  The level of privacy requested by a user could determine how it can be enforced.  It could be enforced through authenticated access, encrypted data transfers, or both.  In most cases, authenticated data access is sufficient.

Private versions of shared and unique data cannot use the same optimization as public data.  For example, the data could not be transferred using a caching daemon if authenticated access is required.   Transferring data unauthenticated, even encrypted, is dangerous due to susceptibility to brute force decryption.



An example policy for a cache that is composed of public shared data could be:

\begin{lstlisting}
Requirements = MY.DiskUsage < TARGET.TotalDisk
ReplicationMethods = "BITTORRENT, DIRECT"
\end{lstlisting}

This policy will allow the CacheD to proactively replicate the caches to available CacheDs.  Further, it will allow the CacheD to replicate using the BitTorrent method, or the Direct method if the remote CacheD does not support or prefer BitTorrent.  

For private shared data, an example policy would look like:

\begin{lstlisting}
Requirements = (MY.DiskUsage < TARGET.TotalDisk) && (TARGET.CacheRequested =?= true)
ReplicationMethods = "DIRECT"
\end{lstlisting}

In this method, the cache would only be replicated to nodes where it is requested.  This will minimize the number of nodes that have this private data stored.  Second, it will use the \texttt{DIRECT} method of distribution, which uses an authenticated and optionally encrypted method of transfer.  

Unique data cannot benefit from caching.  But it can benefit from faster transfers.  As we have seen in Chapter \ref{chapter:campusdatadistribution}, even when caching is not turned on, the unique transfer methods that the CacheD uses can benefit the stage-in time for this unique data.  It is possible to package many jobs worth of unique data into a cache and transfer the cache for each job.  Then, on subsequent execution that require unique data from the package, it will be immediately available from the CacheD's cache.

Packaged unique private data can be transferred with the same policies as shared private data.  Unique public data can be transferred with shared public policies.

\subsection{Policies Utilizing Extended Attributes}

Since ClassAds are schema-less and extendable, attributes can be added that can help in matching.

One example policy is to ban BitTorrent at certain OSG sites.  This policy is useful if sites have policies against certain transfer methods.  While running our experiments from the previous chapter, we were contacted by the administrators of the clusters at Brookhaven National Lab (BNL).  They discovered that we were running BitTorrent on their clusters, contrary to their policy.  The following policy would only use the Direct transfer method at BNL.

\begin{lstlisting}
Requirements = (MY.DiskUsage < TARGET.TotalDisk) && (TARGET.CacheRequested =?= true)
ReplicationMethods = ifThenElse(regexp(".*.bnl.gov$", TARGET.Name), "DIRECT", "BITTORRENT,DIRECT")
\end{lstlisting}

The \texttt{Requirements} are similar to previous policies.  The \texttt{ReplicationMethods} uses the \texttt{ifThenElse} ClassAd function in order determine which replication method to use.  It uses a regular expression to determine if the target CacheD is from BNL.  It tests if the \texttt{Name} ends with the domain name of the \texttt{bnl.edu}, the domain for BNL.  If it does match, then the Direct method is chosen.  If the \texttt{Name} does not match \texttt{bnl.edu}, then both BitTorrent and Direct methods are allowed.

Administrators can add custom attributes using the CacheD Cron.  For example, administrators could add attributes that list the organizations that own the storage resources.  When this attribute is available, both the CacheD and the caches can make policies that will use it.

An example CacheD policy:
\begin{lstlisting}
CacheDOwners = "CMS,HCC"
Requirements = (MY.TotalDisk > TARGET.DiskUsage) && stringListIMember(TARGET.CacheOwner, MY.CacheDOwners)
ReplicationMethods = "BITTORRENT,DIRECT"
\end{lstlisting}

In this policy, the CacheD would only accept replications of caches which have the attribute \texttt{CacheDOwners} and it is set to either \texttt{CMS} or \texttt{HCC}.  There is no method of enforcing the requirement that only members of the CMS or HCC organizations have this have this attribute; therefore, you must create a trust relationship with CacheDs that are allowed to communicate.

A cache policy that would match this policy is:
\begin{lstlisting}
CacheOwner = "HCC"
Requirements = (MY.DiskUsage < TARGET.TotalDisk)
ReplicationMethods = "BITTORRENT,DIRECT"
\end{lstlisting}

Further, a cache could set a policy to only replicate to resources that are owned by the same organization that they belong:
\begin{lstlisting}
CacheOwner = "HCC"
Requirements = (MY.DiskUsage < TARGET.TotalDisk) && stringListIMember(MY.CacheOwner, TARGET.CacheDOwners)
ReplicationMethods = "BITTORRENT,DIRECT"
\end{lstlisting}



%\subsection{Ranking Storage}
%In order to find the most ideal resource for a job, the resources need to be ranked.  The simplest is a greedy approach where the resources are simply ranked by their benchmark speeds.  Additionally, they should only be ranked on the attributes requested by the job, i.e. if the job is only requesting X iops, then only rank resources on the IOPS available.

%It is not immediately clear how the ranking should work.  If we assume that the user accurately describes their application needs, then we can pack the jobs onto resources by placing the job on the resource that meets the IOPS requirements, but has the least amount of IOPS remaining.  This will be an area of research to compare scheduling techniques on execution resources when considering their storage capabilities.


%\section{Data Movement}
%We will consider three different types of data.  The input data, output data, and the job sandbox.  The job sandbox is the environment from which the job will run.  The sandbox is important since the user designs their job to run in this sandbox, and it must be maintained in order for the job to run.  Also, the sandbox is shared input data that multiple executions of the job can utilize.

%The sandbox is a set of files that must be present when the job begins execution.  For example, a sandbox may contain:
%\begin{itemize}
%\item The executable that the job will run.
%\item Libraries necessary for the executable to properly function.
%\item Shared input files such as parameter files or calibration data.
%\end{itemize}

%Some input data could be unique per job, therefore will be considered separately from the job sandbox.  Shared data between many executions can benefit from caching, where unique input data cannot benefit from caching.

%Data for each job can be categorized as either shared, unique, private shared, or private unique.

%\subsection{Categorizing Data}
% Describe shared data



 

% May go into introduction
%Many optimizations may be done to transfer shared data.  For example, people have used caching \cite{blumenfeld2008cms} the shared data per site.  Others have experimented using group transfer protocols such as Bittorrent \cite{cohen2008bittorrent} to distribute the shared data \cite{wei2005collaborative}.

% unique data



%After finding a resource to run on, the job sandbox and input data must be transferred to the remote host.  In order to do this, the remote execution host and the submitter must negotiate how to get the data there.  For example, does the remote host have access to the same NFS server?  Can it mount it?

%\section{Description for these items}
%Logically, we can separate these items into 2 categories
%\begin{itemize}
%\item Requirements for the application
%\item Acceptable methods of data movement
%\end{itemize}

%The users must specify these items in the description of their jobs.  No consensus language for these specifications currently exists.  

%The language for the requirements will be similar to the current specifications for memory and cpu.  The user will request certain storage parameters, and machines will need to provide these metrics just as they do now with cpu and memory.

%The acceptable methods for data movement can either be specified by the user, or by the submitting system.  The system can stage the data to a third party, which will then be used for the transfer to the execution target.  This can be especially useful if multiple jobs use the same input data, a useful example of this is HCC�s use of LVS \cite{zhang2000linux} to serve common files on the OSG.  The server could automatically choose to use HTTP to transfer the files, especially since there are many common files, and the files would be cached on the remote sites using normal HTTP proxy caches.

%Another possible scenario is when starting a job on Amazon EC2.  If it is a virtual machine job, then input data could be created as a CD drive, or a block device, and input into the machine using the block device as input storage.

%\section{Policy language for matchmaking storage}
%The goal is to enable the user to describe their application to the scheduler in such a way that the scheduler can make intelligent decisions on:
%\begin{itemize}
%\item If the application can run on the pool
%\item Where is the ideal location for the job to run
%\item How to get the data to and from the application
%\end{itemize}

% An example policy language for a job is:

\section{User Scenario}
%TODO: work on user scenario

In order to illustrate a user's experience with the policy language, we will use the BLAST example that is used in the previous chapter's experiments.  In this example, we have three types of data:

\begin{description}
	\item[BLAST Database:] A shared public database which will be used by every execution of the job.  The database is widely distributed; therefore, there is no private data to hide.
	\item[BLAST Executables:] Public executables that are originally from the NIH website.  They will be used by every job in the workflow.
	\item[Query Files:] Multiple small files that may not be public.  Each query file is used only by one job.
\end{description}

The BLAST database was shown to be optimally distributed in Chapter \ref{chapter:campusdatadistribution} with the CacheD and BitTorrent.  Therefore, the user would first create the cache, then give it the policy to be proactively replicated to caches with BitTorrent.

\pagebreak

\begin{lstlisting}
$ importCache nrdb <path/to/db>/nrdb*
$ setReplicationPolicy "MY.DiskUsage < TARGET.TotalDisk" "BITTORRENT,DIRECT"
\end{lstlisting}


Next, the BLAST executables are much smaller than the BLAST database, but they are used by every execution.  Therefore, they would benefit from caching and we will use the CacheD.  Since executables are small, the replication will not take long, therefore is no need to proactively replicate the executables.  But, there is nothing private in the BLAST executables, therefore they will be distributed over BitTorrent or Direct method.

\begin{lstlisting}
$ importCache blast_executables <path/to/exes>/blast*
$ setReplicationPolicy "(MY.DiskUsage < TARGET.TotalDisk) && (TARGET.CacheRequested =?= true)" "BITTORRENT,DIRECT"
\end{lstlisting}

Finally, the BLAST queries may be private.  The queries are stored in many small files.  The queries may or may not be optimally transfered and managed by the CacheD.  If the BLAST jobs are short, then the submission node could constantly be transferring the query files to resources that need them.  If the number of running jobs is significant, then this could become a bottleneck.  Using the CacheD, the researcher could avoid significant transfers from the submit host.  By grouping all the queries into a single cache, then transferring that cache for each job, any other jobs on that same node would request the file from the cache rather than the submission node.  If the jobs are short, then this cache would be used often, at the cost of transferring all of the queries the first time.

Since the queries are private, the CacheD must transfer them with the Direct method so that the transfers are authenticated and encrypted.  Further, the CacheD must only replicate them to the nodes that actually request them, so as to not expose the queries on more nodes than necessary.

\begin{lstlisting}
$ importCache blast_executables <path/to/exes>/blast*
$ setReplicationPolicy "(MY.DiskUsage < TARGET.TotalDisk) && (TARGET.CacheRequested =?= true)" "DIRECT"
\end{lstlisting}


%A user creates their submit file and specifies their data.  The above policies will be matched to syntax in the submit file.  The syntax is shown in Listing \ref{lst:inputsyntax}.  We illustrate a shared, non-private case using a BLAST database, and a unique private file with the BLAST queries in Listing \ref{lst:blastsyntax}.  

%\begin{figure}[h!]
%\centering
%\begin{lstlisting}[frame=single,caption={Input Syntax},captionpos=b,label={lst:inputsyntax}] 
%{shared|unique}_{public|private}_input = <file1>,<file2>,...
%\end{lstlisting}
%\end{figure}



%\begin{figure}[h!]
%\centering
%\begin{lstlisting}[frame=single,caption={Blast input syntax},captionpos=b,label={lst:blastsyntax}]
%shared_public_input = blast_database.fasta
%unique_private_input = queries
%\end{lstlisting}
%\end{figure}


%The blast database is public, so there is no need to encrypt or authenticate access to the database.  Further, the database is shared between all executions of the job. The \texttt{queries} may contain personal identifiable information, and are therefore private and need authenticated access control in order to access the data.  

%When the job begins, it will be guaranteed to have the files \texttt{blast\_database.fasta} and \texttt{queries} available to it.  The job framework will decide on the method of transfer and transient storage based on negotiation between the user specified syntax, the worker node, and the submit node.


%\section{Defining Storage Target}
%In this section, we define storage based on it�s capabilities:
%\begin{itemize}
%\item The total space available for an application or set of applications to store data.
%\item The bandwidth available to the storage target.
%\item The IOPS available to read / write to the storage (more applicable to local storage).
%\item Access Protocol
%\end{itemize}

%Therefore, when mentioning storage, we must specify or estimate or discover all of these attributes.


\section{Conclusions}

In a distributed computing system, independent agents are designed to act on behalf of entities such as users, hosts, or entire clusters.  A policy language must exist so that the entities can express their goals to the agents.  

In this chapter, we have designed a policy language based on the HTCondor \mbox{ClassAds} that can be used for expressing policy in data distribution.  We have identified three interaction points for caching agents and designed the semantics for their interaction.

Additionally, we described different input data scenarios and possible replication policies for them.  We gave a recommended policy configuration for the popular BLAST application, including explanations on the policy configuration.  We also included the commands to create and set the policy configuration.

The policy language has been implemented in the CacheD, where it has been tested in Chapter \ref{chapter:campusdatadistribution}.  Further, we illustrated a user scenario of setting this policy language.







%\section{Another Method for Data Transfer}
%In addition to the above methods for transferring data to remote worker nodes, and specifying storage parameters, we can also provide another method for getting data to worker nodes that will better fit the current state of clusters and cyberinfrastructure.  The proposed methods is a dynamic deployment of a storage federation.  This can be done across a single cluster, across many clusters, or over an entire national infrastructure such as the OSG.

%This new method for data transfer relies on peer to peer transfers.  Data is transferred from it�s peer rather than from a single host.  As with all peer to peer systems, the benefits from this method include decreasing the required bandwidth from any single source.  As well as lower latency transfers.


\chapter{Conclusion}
\label{chapter:coordinatingstorage}

% re-iterate the introduction
In this dissertation we optimized distributed computing workflows on a campus grid.  We were interested in optimizing a researcher's use of the computational and storage resources on the campus to increase the reliability and decrease the time to solution for scientific results.  We first extended prior work to enhance the computational capabilities of researchers on a campus.  We then expanded our work to the data needs of modern workflows on the campus.

% Conclusion of BOSCO
Bosco is used to effortlessly create a remote submission endpoint on a cluster without requiring the administrator to install any software.  Bosco is a remote submission framework based upon HTCondor.  It uses the SSH protocol to submit and monitor remotely submitted jobs.  Additionally, it performs file transfers using the same SSH connection.

Improving the user experience was a primary goal of Bosco.  We addressed the user experience by improving the interaction with the user during the installation / configuration.  Another problem area we found is when a user must debug issues with distributed software.  In order to address this, we created a traceroute like utility.  The traceroute utility tests every step of the job submission process, from network access to a properly configured remote scheduler.  If an error is found at any step of the traceroute, a useful message is given to the user, including possible steps to fix the problem.

Bosco and the Campus Factory combine to make an easy to use framework that can distribute jobs to many computational clusters on a campus.  Users are able to effectively distribute their processing to multiple clusters using this framework.  I showed that Bosco transparently and effectively distributes computational jobs across multiple clusters on a campus, while maintaining simple usage for users.

Bosco's usage has increased since I originally published the Bosco paper.  For example, it is heavily used by the University of Chicago in order to submit OSG processing to opportunistic resources around the country.  They find Bosco useful since it does not require the installation of any software on the remote cluster.  Additionally, it has been used in several publications by the CMS experiment when they have used opportunistic resources for data processing.

% Conclusion of the CacheD
For data distribution on the campus, we have presented the HTCondor CacheD, a framework to decrease the stage-in time for large shared input datasets.  Our experiments proved that the CacheD decreases stage-in time for these datasets.  Additionally, the transfer method that the CacheD used can significantly affect the stage-in time of the jobs.

The BitTorrent transfer method proved to be a efficient method to transfer caches from the originator to the execution hosts.  In fact, the transfer time for jobs did not increase as the number of distinct nodes requesting the data increased.  Any bottlenecks that surround the cluster are therefore irrelevant using the BitTorrent transfer method.  In addition, we found that the CacheD using BitTorrent transfer method out performed the popular HTTP transfer method on the Open Science Grid.

% Conclusion of the Policy Language
In a distributed computing system, independent agents are designed to act on behalf of entities such as users, hosts, or entire clusters.  A policy language must exist so that the entities can express their goals to the agents.  

I have designed a policy language based on the HTCondor ClassAds that can be used for expressing policy in data distribution.  I have identified three interaction points for caching agents and designed the semantics for their interaction.  The policy language has been implemented in the CacheD, where it has been tested in Chapter \ref{chapter:campusdatadistribution}. 

Through Bosco, I have created a framework for easy-to-use job submission on the campus.  Through the CacheD, I have created an efficient data distribution agent for the campus.  And through a policy language I created, storage agents can negotiate and represent the users and resource owners goals.  This has created a comprehensive picture for campus computing.

\section{Future Work}

\subsection{Debugging}
% Improved debugging for Bosco
Debugging in distributed computing has always been a challenge.  Many different systems working together can create barriers for message and error propagation.

Debugging in Bosco has always been difficult.  To alleviate some of the debugging burden, we created the \texttt{traceroute} utility described in Section \ref{sec:boscotraceroute}.  But that is not enough to solve every issue.  See Figure \ref{fig:boscojobsubmitflowconclusion} for the flow of job submission in Bosco.

\begin{figure}[h!t]
	\centering
	\includegraphics[width=0.3\textwidth]{images/JobSubmitFlow.pdf}
	\caption{Bosco Job Submission Flow From Submission Host to Remote Cluster}
	\label{fig:boscojobsubmitflowconclusion}
\end{figure}

In Figure \ref{fig:boscojobsubmitflowconclusion}, the Bosco job submission starts at the user, and goes through 6 daemons before the submission reaches the Slurm Scheduler on the remote cluster.  An error can occur at each step in this process.  The error propagation must propagate back to the user if there is an error.  But, propagating the error is very difficult, since each daemon has different methods of internal error propagation.  For example, the GridManager can propagate an error back to the SchedD through HTCondor ClassAds.  But, the Blahp's shell portion cannot propagate an error back except through the linux standard error.

Further work need to be done, and parts of the Bosco submission chain need to be modified or refined in order to enable improved error propagation

\subsection{Flexible Transfer Types}
% Addition of transfer types, and flexible system for defining transfer methods

The CacheD currently has support for two transfer methods, the Direct and the BitTorrent methods.  The BitTorrent method uses the libtorrent library directly for transfers.

The Direct method uses HTCondor's file transfer service.  The file transfer service is expandable through file transfer plugins.  The CacheD would benefit from enabling similar file transfer plugins in order to allow proper negotiation of file transfer methods.

A possible solution is to build off of the file transfer plugins method.  An executable advertises it's transfer capabilities to the CacheD, which in turn uses those capabilities to negotiate with other CacheDs for transferring caches.


\subsection{Co-Scheduling of Data and Computation}

% Scheduling jobs with CacheD
Scheduling of jobs with the caching data could optimize job placement.  Currently, there is no knowledge of the cache placement when scheduling a job.  If the scheduler knew where replicas of the cache were located, it could schedule the jobs run where the replicas are located.  Running the job near the cache will eliminate stage-in time.

Currently, the CacheD reports each cache stored locally to the cache's origin. The origin CacheD keeps a data structure of the locations of the cache replicas.

The job needs to specify the caches are required for the job to run.  The scheduler would then call out to the origin CacheD to see if the cache is located at the target node.  This could be done using the HTCondor ClassAd library plugins.


% Not only where the cache is, but with the policy language, we can guess where the job will be.



% optimize black holes for bittorrent




%% backmatter is needed at the end of the main body of your thesis to
%% set up page numbering correctly for the remainder of the thesis
\backmatter

%% Start the correct formatting for the appendices
\appendix

%% Appendices go here (if you have them)

%% Bibliography goes here (You better have one)
%% BibTeX is your friend

\bibliographystyle{plain}
\bibliography{DerekWeitzelDissertation}

%% Index go here (if you have one)
\end{document}

\endinput
%%
%% End of file `skeleton.tex'.
